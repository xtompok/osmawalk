\chapter*{Závěr}
\addcontentsline{toc}{chapter}{Závěr}

\section*{Zhodnocení}
\addcontentsline{toc}{section}{Zhodnocení}
Úlohou práce bylo navrhnout algoritmy a datové struktury pro odhady pěších tras.
Pro vypracování práce bylo nutné detailně pochopit způsob mapování v~projektu
OpenStreetMap a význam jednotlivých značek. Jako hlavní podklad sloužily wiki
stránky projektu, ale na některé zvyklosti bylo potřeba se zeptat přímo komunity na
diskusních fórech. 

Při následném zpracování se objevily problémy v~nekonzistentním mapování
v~různých částech mapy a chyby v~datech, které jsme museli korektně ošetřit.
Výsledné algoritmy se s~problémy umí vypořádat a ukázková aplikace na ně pouze
upozorní a pokračuje ve zpracování. 

Zjednodušování a předzpracování dat se osvědčilo, výsledná data zachovávají
přesnost a aplikace běží rychleji. Významným přínosem bylo použití spojek, které
suplují chybějící vazby chodníků na silnice i chodníků mezi sebou. Díky spojkám
můžeme plánovat cesty i přes území, kde jsou chodníky zmapovány neúplně a není
jich mnoho. 

Zkratky přes průchozí prostranství se v~současné implementaci příliš neosvědčily.
V~městském prostředí se nevyskytuje mnoho ploch, které by byly průchozí a
zároveň nebyly v~prudkém svahu a nevedla přes ně alespoň nějaká síť
zmapovaných cest a pěšin. Mimo město je většina volných ploch neoznačena, proto
o~nich nemůžeme říci, že jsou průchozí. Též lesy nemusí být obecně průchozí a typy lesů
nebývají v~mapách zaneseny. 

Problémem, který v~malé míře přetrvává i v~současné verzi, je ošetření
okrajových případů při zametání roviny. V~některých situacích nejsou
průsečíky korektně detekovány a vyskytují se spojky vedoucí skrz budovy.  

Vytvořili jsme jednoduchou aplikaci pro hledání v~připravených datech. Tato
aplikace byla určena pro demostraci hledání ve vytvořených datech, proto má jen
základní funkce. Vyhledávání tras i přes použití jednoduchého Dijkstrova
algoritmu je dostatečně rychlé, aby nezdržovalo práci s~aplikací. 

\section*{Výsledky}
\addcontentsline{toc}{section}{Výsledky}
Naše aplikace dává dobré výsledky pro vyhledávání pěších tras i přesto, že data
OSM nejsou na vyhledávání pěších tras vytvářena. Dosažené výsledky jsou lepší
než dávají známé mapové servery a v~dobře zmapovaných oblastech srovnatelné
s~aplikací OsmAnd využívající data OSM. V~oblastech, které nejsou dostatečně
kvalitně zmapovány umí naše aplikace oproti aplikaci OsmAnd využívat i přímo
neuvedené cesty.  Nalezené trasy však stále vyžadují kontrolu kvůli vyskytujícím
se chybám ve spojkách a zkratkách.  Odhadovaná vzdálenost a čas dle zkušeností
autora odpovídají skutečnosti. 


\section*{Náměty pro další rozvoj}
\addcontentsline{toc}{section}{Náměty pro další rozvoj}
Pro další rozvoj aplikace by bylo nutné odstranit nedostatky implementace
zametání roviny, abychom mohli více důvěřovat nalezeným cestám. Poté je možné
zlepšovat aplikaci mnoha způsoby. Jednou možností je zjednodušování grafu pro
úsporu paměti a zkrácení času prohledávání. Jinou možností je zkusit jiné
algoritmy pro vytváření spojek a zkratek a zkoumat vliv na nalezené spojky a
použitelnost zkratek.

Náměty pro rozšíření máme i pro vyhledávací část. Pokud bychom ukládali při
tvorbě dat informace o~zmapovaných přechodech, mohli bychom například při
vyhledávání preferovat přecházení silnice po přechodu. Bylo by také možné
zvětšit možnosti konfigurace sklonů. 

Specifickým tématem je rozšíření vyhledávání mimo město. Námi použité algoritmy
je možné použít i v~mimoměstském prostředí, ovšem problém nastává se zdrojovými
daty. Abychom mohli dobře odhadovat zkratky například přes les, potřebovali
bychom mít mnohem podrobnější data, než OSM poskytuje, například z~důvodu
rozdílné průchodnosti rozdílných typů lesního porostu. Venkov také bývá hůře
zmapovaný, například pole nebývají nijak označená a v~mapách často chybějí údaje
o~oplocení pozemků. Proto by pravděpodobně existovalo mnoho tras korektních
v~mapě, avšak nepoužitelných ve skutečnosti.
