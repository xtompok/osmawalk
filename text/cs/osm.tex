\chapter{Vstupní data}
\section{Projekt OpenStreetMap}
\subsection{O projektu}
Projekt OpenStreetMap\cite{osmweb} vznikl v Anglii v roce 2004 a jeho prvotním cílem bylo
vytvořit volně dostupná geografická data pro Velkou Británii. Iniciativa se
postupně rozrostla do celého světa a dnes mapu pomáhá tvořit přes milion
dobrovolníků. Česká republika je dnes poměrně kvalitně pokryta a zvláště velká
města mají dostatečně detailní pokrytí i pro vyhledávání pěších tras.
\subsection{Datová primitiva TODO: Češtin být můj kamarád} 
Projekt OpenStreetMap používá tři základní geografická primitiva: uzly, cesty a
relace. Ke každému z těchto primitiv mohou být přiřazeny atributy, což jsou
dvojice klíče a hodnoty. 
U polohových dat se neukládá výška, výsledná mapa je pouze dvourozměrná.
Každé primitivum má jedinečné id.
Nyní popíšeme jednotlivá primitiva

{\em Uzly} reprezentují body. Každý bod má určené souřadnice a může, ale nemusí,
mít atributy.

{\em Cesty} reprezentují linie a plochy. Cesta je definována posloupností uzlů.
Pokud má být cesta plochou, musí být první a poslední prvek této posloupnosti
stejný, ale ne každá uzavřená cesta je plocha. Tento problém rozebíráme níže.
Cesty většinou mají atributy, které říkají, co daná cesta reprezentuje. Jedna
cesta také může reprezentovat více fyzických objektů (například silnici s
tramvajovou tratí, park s oplocením).

{\em Relace} reprezentují složtější objekty. Pro náš účel jsou významné relace
typu multipolygon, kterými se reprezentují složitější plochy (nesouvislé, s
dírami). Každá relace obsahuje seznam prvků, ze kterých se skládá. Prvky také
mohou mít uvedenu roli, kterou v dané relaci mají, například u multipolygonů,
zda jsou vnitřní či vnější okraj. Opět mívají atributy určující jejich typ.
Relacemi se dále reprezentují územní hranice, trasy linek MHD, cyklotrasy a
jiné.

\subsection{OSM XML}
Nejobvyklej


\section{Výšková data}
Abychom mohli správně odhadnout náročnost pěší trasy, musíme znát i informace o
nadmořské výšce jednotlivých bodů. Stejně tak jako většina jiných projektů jsme
použili data SRTM\cite{srtmweb}, což jsou volně dostupná výšková data pro celý
svět. % TODO Přepsat!  
