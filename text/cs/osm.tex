\chapter{Vstupní data}
\section{Projekt OpenStreetMap}
\subsection{O projektu}
Projekt OpenStreetMap\cite{osmweb} vznikl v Anglii v roce 2004 a jeho prvotním cílem bylo
vytvořit volně dostupná geografická data pro Velkou Británii. Iniciativa se
postupně rozrostla do celého světa a dnes mapu pomáhá tvořit přes milion
dobrovolníků. Česká republika je dnes poměrně kvalitně pokryta a zvláště velká
města mají dostatečně detailní pokrytí i pro vyhledávání pěších tras.
\subsection{Datová primitiva TODO: Češtin být můj kamarád} 
Projekt OpenStreetMap používá tři základní geografická primitiva: uzly, cesty a
relace. Ke každému z těchto primitiv mohou být přiřazeny atributy, což jsou
dvojice klíče a hodnoty. 
U polohových dat se neukládá výška, výsledná mapa je pouze dvourozměrná.
Každé primitivum má jedinečné id.
Nyní popíšeme jednotlivá primitiva

{\tuc Uzly} reprezentují body. Každý bod má určené souřadnice a může, ale nemusí,
mít atributy.

{\tuc Cesty} reprezentují linie a plochy. Cesta je definována posloupností uzlů.
Pokud má být cesta plochou, musí být první a poslední prvek této posloupnosti
stejný, ale ne každá uzavřená cesta je plocha. Tento problém rozebíráme níže.
Cesty většinou mají atributy, které říkají, co daná cesta reprezentuje. Jedna
cesta také může reprezentovat více fyzických objektů (například silnici s
tramvajovou tratí, park s oplocením).

{\tuc Relace} reprezentují složtější objekty. Pro náš účel jsou významné relace
typu multipolygon, kterými se reprezentují složitější plochy (nesouvislé, s
dírami). Každá relace obsahuje seznam prvků, ze kterých se skládá. Prvky také
mohou mít uvedenu roli, kterou v dané relaci mají, například u multipolygonů,
zda jsou vnitřní či vnější okraj. Opět mívají atributy určující jejich typ.
Relacemi se dále reprezentují územní hranice, trasy linek MHD, cyklotrasy a
jiné.

\subsection{OSM XML}
Nejobvyklejší způsob ukládání OSM dat je ve formátu XML. Všechna primitiva mají
jako jeden z atributů \verb|id|, který značí jejich jednoznačný identifikátor.
Každý typ primitiva má svou vlastní číselnou řadu. V OSM XML jsou
všechna primitiva uložena v následujícím pořadí:

{\tuc Hlavička souboru.} V XML hlavičce je určeno kódování UTF-8 a následně je
otevřen element \verb|osm|, jehož atributem je také použitá verze API. 

{\tuc Uzly.} Zde jsou za sebou postupně vypsány všechny uzly. Uzly jsou
reprezentovány elementem \verb|node|, každý má atribut \verb|lat| a \verb|lon|
určující jejich zeměpisnou šířku a délku, používá se elipsoid WGS-84. Dále mohou
mít vnořené elementy \verb|tag|. Každý tento element má atribut \verb|k|
reprezentující klíč a \verb|v| reprezentující hodnotu pro atributy uzlu.

{\tuc Cesty.} Zde jsou za sebou vypsány cesty. Cesty jsou reprezentovány
elementem \verb|way|, v němž jsou vnořeny elementy \verb|nd|. Každý element
\verb|nd| reprezentuje jeden uzel na cestě, identifikátor tohoto uzlu je uložen
v atributu \verb|ref|. Také cesty mohou mít vnořené elementy \verb|tag|.

{\tuc Relace.} Jako poslední jsou vypsány všechny relace, reprezentované
elementem \verb|relation|. Prvky relací jsou reprezentovány elementy
\verb|member|, které obsahují atributy \verb|type| určující, o jaký typ
primitiva jde, \verb|ref| určující identifikátor tohoto primitiva a \verb|role|
určující roli daného prvku v relaci. I relace mohou mít vnořené elementy
\verb|tag|.

Toto pořadí umožňuje proudově zpracovávat XML, protože když zpracováváme cesty,
tak již máme v paměti všechny uzly, na které se cesty mohou odkazovat, obdobně
s relacemi. 

OSM XML se dá volně stáhnout, většinou jsou tyto soubory pro úsporu místa
zaarchivované a aktualizují se jednou denně. Protože většina aplikací
nepotřebuje data z celého světa, jsou k dispozici i data pro jednotlivé
kontinenty a státy, někdy i s
historií\footnote{\url{http://osm.kyblsoft.cz/archiv/}}. 


\section{Výšková data}
Abychom mohli správně odhadnout náročnost pěší trasy, musíme znát i informace o
nadmořské výšce jednotlivých bodů. Stejně tak jako většina jiných projektů jsme
použili data SRTM\cite{srtmweb}, což jsou volně dostupná výšková data pro celý
svět. % TODO Přepsat!  
