\chapter{Projekt OpenStreetMap}
\section{O projektu}
Projekt OpenStreetMap\cite{osmweb} vznikl v Anglii v roce 2004 a jeho prvotním cílem bylo
vytvořit volně dostupná geografická data pro Velkou Británii. Iniciativa se
postupně rozrostla do celého světa a dnes mapu pomáhá tvořit přes milion
dobrovolníků. Česká republika je dnes poměrně kvalitně pokryta a zvláště velká
města mají dostatečně detailní pokrytí i pro vyhledávání pěších tras.

\section{Datová primitiva} 
Projekt OpenStreetMap používá tři základní geografická primitiva: uzly, cesty a
relace. Ke každému z těchto primitiv mohou být přiřazeny atributy, což jsou
dvojice klíče a hodnoty. 
U polohových dat se neukládá výška, výsledná mapa je pouze dvourozměrná.
Každé primitivum má jedinečné id.
Nyní popíšeme jednotlivá primitiva

{\tuc Uzly} reprezentují body. Každý bod má určené souřadnice a může, ale nemusí,
mít atributy.

{\tuc Cesty} reprezentují linie a plochy. Cesta je definována posloupností uzlů.
Pokud má být cesta plochou, musí být první a poslední prvek této posloupnosti
stejný, ale ne každá uzavřená cesta je plocha. Tento problém rozebíráme níže.
Cesty většinou mají atributy, které říkají, co daná cesta reprezentuje. Jedna
cesta také může reprezentovat více fyzických objektů (například silnici s
tramvajovou tratí, park s oplocením).

{\tuc Relace} reprezentují složtější objekty. Pro náš účel jsou významné relace
typu multipolygon, kterými se reprezentují složitější plochy (nesouvislé, s
dírami). Každá relace obsahuje seznam prvků, ze kterých se skládá. Prvky také
mohou mít uvedenu roli, kterou v dané relaci mají, například u multipolygonů,
zda jsou vnitřní či vnější okraj. Opět mívají atributy určující jejich typ.
Relacemi se dále reprezentují územní hranice, trasy linek MHD, cyklotrasy a
jiné.

\section{OSM XML}
Nejobvyklejší způsob ukládání OSM dat je ve formátu XML. Všechna primitiva mají
jako jeden z atributů \verb|id|, který značí jejich jednoznačný identifikátor.
Každý typ primitiva má svou vlastní číselnou řadu. V OSM XML jsou
všechna primitiva uložena v následujícím pořadí:

{\tuc Hlavička souboru.} V XML hlavičce je určeno kódování UTF-8 a následně je
otevřen element \verb|osm|, jehož atributem je také použitá verze API. 

{\tuc Uzly.} Zde jsou za sebou postupně vypsány všechny uzly. Uzly jsou
reprezentovány elementem \verb|node|, každý má atribut \verb|lat| a \verb|lon|
určující jejich zeměpisnou šířku a délku, používá se elipsoid WGS-84. Dále mohou
mít vnořené elementy \verb|tag|. Každý tento element má atribut \verb|k|
reprezentující klíč a \verb|v| reprezentující hodnotu pro atributy uzlu.

{\tuc Cesty.} Zde jsou za sebou vypsány cesty. Cesty jsou reprezentovány
elementem \verb|way|, v němž jsou vnořeny elementy \verb|nd|. Každý element
\verb|nd| reprezentuje jeden uzel na cestě, identifikátor tohoto uzlu je uložen
v atributu \verb|ref|. Také cesty mohou mít vnořené elementy \verb|tag|.

{\tuc Relace.} Jako poslední jsou vypsány všechny relace, reprezentované
elementem \verb|relation|. Prvky relací jsou reprezentovány elementy
\verb|member|, které obsahují atributy \verb|type| určující, o jaký typ
primitiva jde, \verb|ref| určující identifikátor tohoto primitiva a \verb|role|
určující roli daného prvku v relaci. I relace mohou mít vnořené elementy
\verb|tag|.

Toto pořadí umožňuje proudově zpracovávat XML, protože když zpracováváme cesty,
tak již máme v paměti všechny uzly, na které se cesty mohou odkazovat, obdobně
s relacemi. 

OSM XML se dá volně stáhnout, většinou jsou tyto soubory pro úsporu místa
zaarchivované a aktualizují se jednou denně. Protože většina aplikací
nepotřebuje data z celého světa, jsou k dispozici i data pro jednotlivé
kontinenty a státy, někdy i s
historií\footnote{\url{http://osm.kyblsoft.cz/archiv/}}. 

\section{Problémy v OSM datech}
Ačkoli jsou data z OSM poměrně kvalitní a přesné, během jejich zpracování jsme
narazili na některá problematická místa. 

Různorodé označování ploch. Tento problém plyne již z definice geografických
primitiv a atributů objektů. Protože jak plochy, tak linie jsou označovány v OSM
stejným primitivem {\em cesta}, musí se při rozlišování hledět na jejich atributy.
Bohužel ani zde není jejich používání sjednoceno. Aby mohla být {\em cesta}
plochou, musí mít první a poslední {\em uzel} stejný. Ale ne každá uzavřená {\em
cesta } je plochou. Například uzavřená silnice se pokládá za kruhový objezd a
nikoli plochu. Aby byla považována za plochu, musí mít nastaven klíč
\verb|area| hodnotu \verb|yes|. Naopak pokud má {\em cesta} například klíč \verb|landuse|
nebo \verb|building|, je za plochu považována, i když nemá klíč \verb|area|.

Nekorektní objekty. I přesto, že editory používané pro úpravu OSM se snaží
hlídat správnost vytvořených objektů, se v mapě vyskytují různé chybně
definované objekty. Nalezli jsme například cesty, které samy sebe kříží,
chybějící elementy v hranicích multipolygonu a jiné. Výhodou OSM je, že jsme je
mohli ihned opravit, ale přesto je musí umět program korektně vyhodnotit.

Více bodů na stejných souřadnicích. Ač by se pro spojování sousedních ploch měly
využívat společné body na hranicích, v některých oblastech jsme našli
problematická místa, kde polovina ploch používala jeden bod a druhá polovina
jiný na stejných souřadnicích. Tyto dva body spolu nebyly nijak propojeny a
pravděpodobně vznikly při některém importu z jiných datových zdrojů. Bohužel
takovéto složité chyby se nedají jednoduše ručně opravit a je opět potřeba je
korektně zpracovat.

Různá sémantika atributů na různých místech. Při výběru atributů, které zahrnout
ke zpracování, jsme se setkali s používáním stejných atributů v různých
významech. Například klíč \verb|landuse| s hodnotou \verb|residential| v
některých místech označoval velké plochy s obytnými budovami, jinde volně
přístupná prostranství mezi těmito budovami. Kvůli těmto nejednoznačnostem jsme
některé atributy nemohli použít, protože by vytvářely v některých částech města
nekorektní výsledky.

Neúplně navazující cesty. Kvalita mapovaných cest v OSM je silně závislá na typu
cesty. Zatímco silnice, které jsou často využívány pro hledání tras, jsou
většinou navzájem navázány správně, chodníky často v datech úplně chybí, nebo
nejsou napojeny na silnice. Rovněž přechody často chybí. Proto v našem programu
zkoumáme i okolí cest a přidáváme možné spojky mezi chodníky a silnicemi.

\section{Využívané informace z OSM}
Protože chceme umět nejen vyhledávat po cestách, ale i vytvářet trasy přes
průchozí prostranství, potřebujeme mimo cest zpracovávat i budovy, prostranství 
a další objekty. Zpracováváme cesty, které popisují fyzické objekty v terénu,
naopak vynecháváme správní hranice, podzemní objekty (např. metro) a cesty
popisující služby a občanskou vybavenost. Uzly zpracováváme jen jako body s
danou polohou, jejich atributy nevyužíváme. Z relací používáme pouze
multipolygon, kterým jsou často reprezentovány budovy.

