\chapter{Příprava dat}
V této kapitole si popíšeme, jak z OSM dat vyrobíme graf pro vyhledávání pěších
tras. Nejprve z dat vybereme pouze ta, která potřebujeme, a uložíme si je do
našeho formátu k dalšímu zpracování. Poté zjednodušíme budovy, rozdělíme příliš
dlouhé cesty, doplníme spojky mezi cestami, připravíme zkratky pro trasy přes
průchozí prostranství a nakonec ze všech dat vytvoříme vyhledávací graf.

\section{Klasifikace OSM dat}
V našem formátu pro přípravu dat jsou cesty a multipolygony rozděleny do
kategorií podle typů objektů, které reprezentují. Kategorie jsou
následující:
\begin{itemize}
	\item \verb|WAY| -- cesta, o které nic nevíme, rezervováno pro speciální
	případy
	\item \verb|WATER| -- strouha či přeskočitelný potok
	\item \verb|PARK| -- plocha s udržovanou zelení, nízkou trávou
	\item \verb|GREEN| -- plocha s neudržovanou zelení, často jsou v ní
	prošlapané nezmapované cestičky
	\item \verb|FOREST| -- les v městském slova smyslu, obvykle bývá pěšky
	průchozí
	\item \verb|PAVED| -- zpevněná cesta či plocha
	\item \verb|UNPAVED| -- nezpevněná cesta či plocha
	\item \verb|STEPS| -- schody
	\item \verb|HIGHWAY| -- silnice, silnice s chodníkem
	\item \verb|BARRIER| -- překážka pro chodce nepřekonatelná, typ se
	nerozlišuje, například budova, řeka, plot
	\item \verb|DIRECT| -- spojka mezi cestami
\end{itemize}
Rozdělení do kategorií probíhá s pomocí konfiguračního souboru, ve kterém můžeme
zvolit jaké atributy mají objekty mít, aby patřily do dané kategorie. Protože
ani informace, zda jde o plochu, není ve specifikaci OSM přesně definována, jsou
pro určení, zda jde o plochu, použity údaje z konfiguračního souboru. Po
rozdělení objektů do kategorií jsou smazány body, které neleží na žádné cestě,
protože takovéto body dále nikde nepotřebujeme.

\section{Převod multipolygonů na cesty}
Multipolygony popisují objekty se složitější strukturou. Každý multipolygon se
skládá z několika cest, přičemž cesty mohou být vnější -- obvodové -- a
vnitřní, pokud má objekt díry. Při správné definici by mělo sjednocení cest
vypadat jako několik kružnic. Takto je také řešen převod multipolygonů na běžné
cesty. Dokud má nějaké vnější cesty, pokusíme se z nich složit kružnici.  Pokud
se stane, že některý segment kružnice chybí, je vypsána chyba a multipolygon se
dále nezpracovává. Stejně tak, pokud kružnice končí v nějakém svém bodě, který
ale není počáteční. 

\section{Spojení budov}
Ve městě jsou velmi časté bloky budov. V OSM datech jsou budovy reprezentovány
obvykle samostatně, přičemž nám stačí znát obvod celého bloku. Proto cesty v
jednotlivých blocích nahradíme jednou cestou reprezentující obvod tohoto bloku.
Nejprve vybereme všechny budovy a vytvoříme si z nich graf sousednosti
jednotlivých budov. Poté v každé komponentě najdeme bod na obvodu a postupně
po obejdeme po obvodu celou budovu a vytváříme při tom novou cestu. Původní
cesty potom smažeme. Když máme zpracované všechny budovy, smažeme body uvnitř
budov, které nyní neleží na žádné cestě.

Při zpracování se mohou vyskytnout dva druhy problémů. Jednak se uvnitř bloků
mohou vyskytovat cesty, v tom případě spojování cest přerušíme, protože bychom
se mohli připravit o možnost přidání spojek mezi cestami uvnitř bloku. Druhý
problém souvisí s problémem více bodů na stejných souřadnicích. Může se stát, že
jsou budovy špatně navázané, potom výsledek může vypadat zvláštně, ale stále se
bude jednat o korektní cestu. Budovy také mohou mít na obvodu dva sousední body
na stejných souřadnicích. Pokud narazíme na tuto situaci, zjednodušování
ukončíme, protože nedokážeme určit, jestli tento bod leží na cestě tvořící
obvod, nebo na cestě vedoucí dovnitř budovy. Tento problém se ale nevyskytuje
příliš často, proto jsme ho speciálně neošetřovali.

\section{Rozdělení dlouhých úseků}
Pro další zpracování se nám bude hodit, abychom cesty, po kterých budeme
plánovat trasy, neměly příliš dlouhé úseky. Proto se na všechny takové cesty
podíváme a v případě, že obsahují úseky delší než zvolená vzdálenost, tyto úseky
rovnoměrně rozdělíme na kratší. 

\section{Spojky mezi cestami}
Jak již bylo zmíněno, chodníky jsou v OSM zmapovány jen některé a často nejsou
na sebe správně navázány. Proto jsme se rozhodli prohledat okolí každého uzlu a
spojit každé dva uzly, které jsou k sobě blíže než 20 m, úsečkou. Při zkoumání
OSM dat jsme zjistili, že mnohé obytné domy kolem sebe nemají značené ploty a
proto jsme vzdálenost zvolili takto nízkou, protože poté je pravděpodobnost, že
jde například o dvě ulice oddělené domky se zahrádkami, poměrně malá.  

Mezi velmi blízkými uzly ale vznikalo příliš velké množství spojek, které by
navíc nepřinášely významné zkrácení trasy, proto jsme u každého uzlu vybrali
spojky s určitou pravděpodobností tak, aby měl každý uzel průměrný stupeň šest.
Tyto spojky navíc mohou vést přes nějakou překážku, jako je plot nebo zeď, proto
potřebujeme zkontrolovat kolize všech nových úseček s překážkami. K tomuto účelu
jsme využili algoritmus zametání roviny TODO: rozepsat, obrázek.  

\section{Zkratky přes průchozí prostranství}
Pokud procházíme ve městě přes větší park nebo náměstí, často nemusíme chodit
jen po chodnících, ale můžeme projít přes prostranství přímo. Přímé cesty přes
průchozí prostranství proto také chceme zanést do vyhledávacího grafu. K jejich
vytvoření jsme využili modifikovaný algoritmus vytváření spojek mezi cestami.
Dvojice bodů v průchozích prostranstvích jsme spojovali úsečkami. Tentokrát jsme
ale hledali dvojice bližší než 300 m, protože parky mohou být poměrně rozlehlé.
Omezení na průměrný stupeň jsme snížili na dva, protože zvýšením vzdálenosti
hledání při vyšším stupni vznikalo příliš mnoho úseček, které vedly podobně a
proto nepřinášely významné zlepšení vyhledané trasy. Protože i na průchozích
prostranstvích se vyskytují překážky, opět jsme pomocí zametání roviny
odstranili všechny úsečky, které procházely přes nějakou překážku.

\section{Vytvoření vyhledávacího grafu}
Poté, co jsme připravili data, jsme vytvořili vyhledávací graf. Protože kolize
přímých cest s překážkami jsme vyřešili již při jejich vytváření, jsou všechny
cesty korektní. Proto se překážkami již dále nezabýváme a pouze z cest vytvoříme
vyhledávací graf. Uložíme seznam všech uzlů, které jsou na cestách použity a
cesty rozdělíme na jednotlivé hrany grafu mezi dvěma uzly. U hran zachováme
údaje o typu cesty, protože je budeme využívat při vyhledávání.
