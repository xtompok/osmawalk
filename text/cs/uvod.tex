\chapwithtoc{Úvod}

Problém hledání cesty ve městě je častý pro většinu lidí. Ve městě se vyskytuje
mnoho různých překážek jako jsou ploty, zábradlí či frekventované silnice,
stejně tak jako mnoho průchozích prostranství, jako jsou náměstí, parky či sady.
Dobrý vyhledávač by se měl umět překážkám vyhnout a měl by také umět využívat
průchozí prostranství. V~naší práci se snažíme ukázat, jak takovéto věci řešit.

Našim cílem je vytovřit vyhledávač spojení ve městě, který bude umět vhodně
kombinovat pěší cesty a cesty městskou hromadnou dopravou. Tato práce by měla
být jedním z~modulů budoucí aplikace a měla by zajišťovat právě vyhledávání
pěších cest.

%\section{Co chci počítat}
Abychom mohli rychle vyhledávat pěší trasy, potřebujeme k~tomu mít mapu vhodně
reprezentovanou. Vstupní mapová data cheme převést do vhodného formátu, který
bude obsahovat všechny potřebné informace, budeme schopni v~něm rychle
vyhledávat a nebudeme muset dělat náročné výpočty. 

%\section{Zdroje dat}
Abychom mohli vyhledávat trasy, potřebujeme k~tomu vhodné mapové podklady.
Protože jsme chtěli, aby bylo možné zpracovaná data volně používat a šířit, 
potřebovali jsme získat i takové mapové podklady. Proto jsme si vybrali jako
zdroj dat OpenStreetMap (OSM), volně dostupné mapové podklady vytvářené komunitou. 
Projekt OpenStreetMap využívá k~tvorbě mapy mimo práce dobrovolníků také jiné
volně dostupné mapové podklady a data tohoto projektu jsou pravěpodobně nejlepší
veřejně dostupná mapová data. Jako zdroj dat o~nadmořské výšce jsme použili data
z~projektu SRTM, která jsou také volně dostupná.

%\section{Co už kdo napsal}
K~porovnání výsledků naší aplikace jsme použili nejznámější webové mapové
aplikace --- Google Maps a Mapy.cz. Také jsme nalezené trasy porovnávali s~jinými
vyhledávači používajícími OSM data --- OsmAnd a TODO. Nalezené trasy jsme také
porovnávali s~vlastní znalostí terénu a skutečně používaných cest.

