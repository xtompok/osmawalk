\chapwithtoc{Úvod}

chceme hledat cestu
chceme chodit pěšky
existují různé služby
většina určena primárně pro auta -> nereflektuje pěší potřeby
 - kros průchozích prostranství


Hledání cesty ve městě je častou situací většiny lidí. Ve městě se vyskytuje
mnoho různých překážek jako jsou ploty, zábradlí či frekventované silnice,
i mnoho průchozích prostranství, například náměstí, parky či sady. V naší práci
se snažíme z mapových dat vytvořit formát vhodný pro rychlé vyhledávání pěších
tras využívajících i průchozí prostranství. Tato práce by měla být jedním z~modulů
budoucí aplikace pro vyhledávání spojení pěšky a MHD po městě. 

%\section{Co chci počítat}
Abychom mohli rychle vyhledávat pěší trasy, potřebujeme k~tomu mít mapu vhodně
reprezentovanou. Vstupní mapová data postupně zpracováváme a vybíráme z nich
použitelné informace. Na konci tohoto procesu vytvoříme graf popisující možné
pěší trasy. V tomto grafu již můžeme vyhledávat pomocí grafových algoritmů pro
hledání cest, v ukázkové aplikaci využíváme Dijkstrův algoritmus. 

%\section{Zdroje dat}
Abychom mohli vyhledávat trasy, potřebujeme k~tomu vhodné mapové podklady.
Protože jsme chtěli, aby bylo možné zpracovaná data volně používat a šířit, 
potřebovali jsme získat i takové mapové podklady. Proto jsme si vybrali jako
zdroj dat OpenStreetMap (OSM), volně dostupné mapové podklady vytvářené komunitou. 
Projekt OpenStreetMap využívá k~tvorbě mapy mimo práce dobrovolníků také jiné
volně dostupné mapové podklady a poskytuje pravěpodobně nejlepší veřejně
dostupná mapová data. Jako zdroj dat o~nadmořské výšce jsme použili data
z~projektu SRTM, která jsou také volně dostupná.

%\section{Co už kdo napsal}
K~porovnání výsledků naší aplikace jsme použili nejznámější webové mapové
aplikace -- Google Maps a Mapy.cz. Také jsme nalezené trasy porovnávali s~jinými
vyhledávači používajícími data OSM -- OsmAnd a TODO. Nalezené trasy jsme také
porovnávali s~vlastní znalostí terénu a skutečně používaných cest.

%TODO: obrazek - vstupni data a vystupni data

V první kapitole se zabýváme zdrojovými geografickými daty. Popíšeme, jakým
způsobem OpenStreetMap popisuje objekty reálného světa a jaké to přináší
problémy při zpracování. Ukážeme jeden ze způsobů, kterým jsou data ukládána -
OSM XML. Popíšeme, které informace z dat OSM využíváme a s jakými problémy jsme
se při jejich zpracování museli vypořádat. Na konci se zmíníme o používaných
výškových datech z projektu SRTM.

Ve druhé kapitole popíšeme používané formáty. Popíšeme souřadný systém UTM a
důvody, proč jsme ho zvolili jako interní reprezentaci souřadnic. Popíšeme
ProtocolBuffery používané jako binární reprezentace našich datových struktur.
Popíšeme datovou strukturu používanou při zpracovávání mapových dat a strukturu
vyhledávacího grafu z dat vzniklého.

Ve třetí kapitole popíšeme použité algoritmy

Ve čtvrté kapitole poíšeme detaily implementace jednotlivých fází zpracování,
popíšeme problémy, které při implmentaci nastaly a jakým způsobem jsme je
řešili. 

V páté kapitule se budeme zabývat hledáním ve vygenerovaných datech. Popíšeme
Dijkstrův algoritmus pro vyhledávání v grafu a popíšeme ukázkovou vyhledávací
aplikaci. 

V šesté kapitole porovnáme výsledky naší aplikace s výsledky jiných vyhledávačů
a vlastními zkušenostmi. 


