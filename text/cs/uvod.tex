\chapwithtoc{Úvod}

chceme hledat cestu
chceme chodit pěšky
existují různé služby
většina určena primárně pro auta -> nereflektuje pěší potřeby
 - kros průchozích prostranství
může to být modul pro MHD


Děláme to takto:


Přikládáme:

%DONE
V první kapitole zmiňujeme motivaci k naší práci a stanovujeme cíl práce.
Podrobnější informace o zdrojích dat a popis jejich formátu uvádíme ve druhé
kapitole. Ve třetí kapitole popisujeme používané formáty a datové struktury.
Použité algoritmy a postup při přípravě dat pro vyhledávání je obsahem čtvrté
kapitoly. Popis implementace celého procesu přípravy vyhledávacích dat a
problémů, které se při implementaci vyskytly uvádíme v páté kapitole. V šesté
kapitole navrhujeme algoritmy na ukázku hledání v datech a popisujeme jejich
implementaci. Zhodnocení nalezených tras a jejich porovnání s trasami jiných
vyhledávačů je předmětem sedmé kapitoly. V závěru zhodnocujeme celou práci a
uvádíme možnosti dalšího rozšíření.


Hledání cesty ve městě je častou situací většiny lidí. Ve městě se vyskytuje
mnoho různých překážek jako jsou ploty, zábradlí či frekventované silnice,
i mnoho průchozích prostranství, například náměstí, parky či sady. V naší práci
se snažíme z mapových dat vytvořit formát vhodný pro rychlé vyhledávání pěších
tras využívajících i průchozí prostranství. Tato práce by měla být jedním z~modulů
budoucí aplikace pro vyhledávání spojení pěšky a MHD po městě. 

%\section{Co chci počítat}
Abychom mohli rychle vyhledávat pěší trasy, potřebujeme k~tomu mít mapu vhodně
reprezentovanou. Vstupní mapová data postupně zpracováváme a vybíráme z nich
použitelné informace. Na konci tohoto procesu vytvoříme graf popisující možné
pěší trasy. V tomto grafu již můžeme vyhledávat pomocí grafových algoritmů pro
hledání cest, v ukázkové aplikaci využíváme Dijkstrův algoritmus. 

%\section{Zdroje dat}
Abychom mohli vyhledávat trasy, potřebujeme k~tomu vhodné mapové podklady.
Protože jsme chtěli, aby bylo možné zpracovaná data volně používat a šířit, 
potřebovali jsme získat i takové mapové podklady. Proto jsme si vybrali jako
zdroj dat OpenStreetMap (OSM), volně dostupné mapové podklady vytvářené komunitou. 
Projekt OpenStreetMap využívá k~tvorbě mapy mimo práce dobrovolníků také jiné
volně dostupné mapové podklady a poskytuje pravěpodobně nejlepší veřejně
dostupná mapová data. Jako zdroj dat o~nadmořské výšce jsme použili data
z~projektu SRTM, která jsou také volně dostupná.

%\section{Co už kdo napsal}
K~porovnání výsledků naší aplikace jsme použili nejznámější webové mapové
aplikace -- Google Maps a Mapy.cz. Také jsme nalezené trasy porovnávali s~jinými
vyhledávači používajícími data OSM -- OsmAnd a TODO. Nalezené trasy jsme také
porovnávali s~vlastní znalostí terénu a skutečně používaných cest.

%TODO: obrazek - vstupni data a vystupni data

