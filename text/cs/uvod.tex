\chapwithtoc{Úvod}

Hledat trasy ve městě či v~přírodě potřebuje každý. Dříve se člověk musel
spolehnout na mapy a vlastní orientační smysl, dnes je možné tento úkol
přenechat technice. V~současné době existují jak internetové služby, tak
samostatné aplikace pro hledání tras. Starší a dnes plně použitelné jsou služby a
aplikace hledající trasy pro automobily. V~poslední době jsou k~dispozici i
vyhledávače pro pěší trasy, ale nalezené trasy obvykle nebývají optimální,
protože kvalita dat je posuzována především z~pohledu autonavigace. Rozhodli
jsme se zkusit upravit dostupná data tak, abychom získali podklady pro
kvalitnější hledání pěších tras. Tohoto cíle se nám podařilo dosáhnout. 

V~první kapitole zmiňujeme motivaci k~naší práci a stanovujeme cíl práce.
Podrobnější informace o~zdrojích dat a popis jejich formátu uvádíme ve druhé
kapitole. Ve třetí kapitole popisujeme používané formáty a datové struktury.
Použité algoritmy a postup při přípravě dat pro vyhledávání jsou obsahem čtvrté
kapitoly. Popis implementace celého procesu přípravy vyhledávacích dat a
problémů, které se při implementaci vyskytly, uvádíme v~páté kapitole. V~šesté
kapitole navrhujeme algoritmy na ukázku hledání v~datech a popisujeme jejich
implementaci. Zhodnocení nalezených tras a jejich porovnání s~trasami jiných
vyhledávačů je předmětem sedmé kapitoly. V~závěru zhodnocujeme celou práci a
uvádíme možnosti dalšího rozšíření.

V~příloze A~popisujeme používání vytvořených programů pro tvorbu vyhledávacích
dat a vyhledávací aplikace. V~příloze B popisujeme obsah přiloženého CD.

%%TODO: obrazek - vstupni data a vystupni data


