\chapter{Formáty}
Našim cílem je vytvořit data ve formátu vhodném na rychlé vyhledávání pěších
tras. K~tomu potřebujeme jednak samotný formát a jednak postup, kterým z~OSM a
SRTM dat vytvoříme kvalitní data pro vyhledávání. Při výběru vhodného formátu
ukládání dat jsme zvolili Protocol Buffery\cite{pbfweb} vyvinuté Googlem. Data
pro vyhledávání budeme ukládat jako graf, jako seznam vrcholů a seznam hran.



\section{Protocol Buffer}
Protocol Buffer je technologie jejímž cílem je vytvořit jednoduchý a rychlý
formát pro ukládání strukturovaných dat, který by byl nezávislý na použitém 
programovacím jazyku a platformě. 

Strukturu dat nejprve popíšeme ve zvláštním souboru pomocí definovaného
jazyka\cite{pbfspec}. Z~něj se následně generují funkce pro použití
v~jednotlivých programovacích jazycích. Základní jednotkou pro přenos dat je
zpráva, která obsahuje datové položky. Každá položka má určenou násobnost, typ a
značku.

\begin{verbatim}

package graph;
import "types.proto";

message Vertex {
    required sint64 idx = 4;
    required sint64 osmid = 1;
    required double lat = 2;
    required double lon = 3;
    optional sint32 height = 5;
    
}

message Graph {
    repeated Vertex vertices = 1;
    repeated Edge edges = 2;
}
\end{verbatim}

{\tuc Násobnost} určuje, kolik prvků může být v~položce uloženo. Možnosti jsou
\verb|required|, pak musí každá zpráva mít v~položce právě jeden prvek,
\verb|optional|, pak ve zprávě tato položka může, ale nemusí mít přiřazen prvek,
\verb|repeated|, pak ve zprávě může být pod položkou uloženo libovolné množství
prvků, i žádný. Tato násobnost fugnuje obdobně jako pole v~programovacích
jazycích. 

{\tuc Typy} jsou podobné typům v~programovacích jazycích, obsahují několik typů celých
čísel, čísla s~plovoucí desetinnou čárkou, řetězce a logické hodnoty. Typem může
být také zpráva, což umožňuje hierarchické strukturování dat. 

{\tuc Jméno} položky slouží pro generování jmen funkcí pro přístup k~dané
položce. Ve vygenerované binární podobě zprávy není obsaženo. 

{\tuc Značka} je jednoznačný číselný identifikátor každé položky, musí být v~rámci
zprávy jedinečná a pokud se formát zprávy v~čase mění, neměly by se značky znovu
používat pro jiné položky.

Dále také může být deklarován výčtový typ, kterým můžeme popisovat prvky
nějaké množiny, je možné vytvářet jmenné prostory pomocí balíčků a používat
definice z~jiných souborů se specifikací.

Když máme specifikovaný formát, necháme si vygenerovat kód pro práci s~tímto
formátem. V~tom je velká výhoda Protocol Bufferů, pomocí kompilátoru se ze
souboru ze specifikací formátu vygeneruje kód v~cílovém jazyce, který nám umožní
pracovat se zprávami jako s~objekty v~objektových jazycích či jako se
strukturami v~C. Takto můžeme s~vytvořenými daty jednoduše pracovat z~mnoha
podporovaných jazyků. Pokud bychom chtěli použít programovací jazyk, který není
podporován, je možné si napsat vlastní dekodér, binární forma Protocol Bufferů
je veřejně zdokumentována.\cite{pbfenc}



\section{Formát pro přípravu dat}
Pro zpracovávání OSM dat jsme navrhnuli vlastní formát, protože formát OSM
obsahuje mnoho údajů, které nepotřebujeme a je tak zbytečně pomalý na zpracování
a naopak některé údaje, které vytváříme během zpracování, by se v~něm obtížně
reprezntovaly. Základní struktura zůstává zachována. Mapu ukládáme jako zprávu
\verb|Map|, která obsahuje seznamy zpráv \verb|Node| pro uzly, \verb|Way| pro
cesty a \verb|Multipolygon| pro multipolygony. Jiné typy relací nejsou
využívány, proto se do tohoto formátu neukládají.

Zpráva \verb|Node| pro uzel obsahuje položky:
\begin{itemize}
	\item \verb|id| -- OSM identifikátor uzlu
	\item \verb|lat| -- zeměpisná šířka v~UTM
	\item \verb|lon| -- zeměpisná délka v~UTM
	\item \verb|height| -- nadmořská výška v~metrech
	%\item \verb|objtype| -- typ uzlu
	\item \verb|inside| -- logická hodnota vyjadřující, zda leží uzel uvnitř
	nějaké překážky, například uzly reprezentující pasáž procházející domem
	%\item \verb|nodeidx| -- pořadí uzlu v seznamu
\end{itemize}
Seznam uzlů se v~průběhu zpracování mění, jsou z~něho mazány uzly, které již
nebudou potřeba a jsou do něj přidávány uzly vzniklé například dělením příliš
dlouhých úseků.

Zpráva \verb|Way| pro cestu obsahuje položky:
\begin{itemize}
	\item \verb|id| -- OSM identifikátor cesty
	\item \verb|refs| -- seznam OSM identifikátorů uzlů, ze kterých se cesta
	skládá
	\item \verb|area| -- logická hodnota určující, jestli je daná cesta
	plochou
	%\item \verb|barrier| -- logická hodnota určující, jestli je daná cesta
	překážkou
	\item \verb|type| -- kategorie reprezentovaného objektu TODO: odkaz
	%\item \verb|bordertype| -- pokud je cesta plocha, jakým způsobem je
	ohraničena
	%\item \verb|crossing| -- jaké kategorie cest cesta kříží
	\item \verb|bridge| -- logická hodnota vyjadřující, zda vede cesta na mostě
	\item \verb|tunnel| -- logická hodnota vyjadřující, zda vede cesta
v~tunelu, průchodu, \dots
	%\item \verb|wayidx| -- pořadí cesty v seznamu
	%\item \verb|render| -- logická hodnota vyjadřující, jestli se bude cesta rednerovat
\end{itemize}
Seznam hran se v~průběhu zpracování také mění, jsou do něj přidávány obrysy
bloků budov a naopak odebírány obrysy jednotlivých budov v~bloku.

Zpráva \verb|Multipolygon| slouží k~uložení těch relací, které jsou
multipolygony. Obsahuje následující položky:
\begin{itemize}
	\item \verb|id| -- OSM identifikátor relace
	\item \verb|refs| -- seznam OSM identifikátorů cest, ze kterých se
	multipolygon skládá
	\item \verb|roles| -- role jednotlivých cest (zda jde o~vnitřní či vnější
	okraj)
	\item \verb|type| -- kategorie reprezentovaného objektu
\end{itemize}
Seznam multipolygonů se v~průběhu zpracování zkracuje, multipolygony jsou
převáděny na cesty a žádné nové vytvářeny nejsou.



\section{Formát vyhledávacího grafu}
Graf ukládáme jako dva seznamy --- seznam vrcholů a seznam hran. U~každého
vrcholu si pamatujeme jeho souřadnice ve formátu UTM, jeho nadmořskou výšku
v~metrech a jeho identifikátor v~OSM. U~hrany si pamatujeme počáteční a koncový
vrchol a její typ značící typ cesty, kterou reprezentuje. Všechny hrany jsou
brány jako obousměrné.

Vyhledávací graf ukládáme jako zprávu \verb|Graph| se dvěma položkami ---
seznamem vrcholů \verb|vertices| a seznamem hran \verb|edges|. Zpráva
\verb|Vertex| popisující vrchol obsahuje položky:
\begin{itemize}
	\item \verb|osmid| -- identifikátor vrcholu v~OSM
	\item \verb|lat| -- zeměpisná šířka v~UTM
	\item \verb|lon| -- zeměpisná délka v~UTM
	\item \verb|height| -- nadmořská výška v~metrech
\end{itemize}
Zpráva \verb|Edge| popisující hranu obsahuje tyto položky:
\begin{itemize}
	\item \verb|vfrom| -- index prvního vrcholu
	\item \verb|vto| -- index druhého vrcholu
	\item \verb|type| -- typ cesty, po které hrana vede
\end{itemize}

