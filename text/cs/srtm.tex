\section{Výšková data}
Abychom mohli správně odhadnout náročnost pěší trasy, musíme znát i informace
o~nadmořské výšce jednotlivých bodů. Stejně tak jako většina jiných projektů jsme
použili data SRTM \cite{srtmweb}, což jsou volně dostupná výšková data pro celý
svět.

Shuttle Radar Topography Mission byl projekt NASA, kdy při letu raketoplánu
Endeavour v~roce 2000 byla pomocí radarové interferometrie změřena velká část
Země a zpracovaná data byla následně poskytnuta volně k~dispozici.\footnote{\url{http://dds.cr.usgs.gov/srtm/version2_1/}} 

Výšková data byla měřena po třech úhlových vteřinách (v~USA po jedné vteřině),
tudíž v~České republice jsou data v~mřížce $90\times60$\,m. Data jsou rozdělena
do tabulek po jednom stupni, přičemž sousední tabulky se vždy jedním sloupcem
nebo řádkem překrývají. Každá tabulka má tedy 1201 sloupců a 1201 řádků, kde
řádky odpovídají zeměpisné šířce a sloupce zeměpisné délce.

Nadmořské výšky jsou kódovány celými 16bitovými čísly v~big-endian udávajícími
nadmořskou výšku v~metrech. Pokud se v~některém místě nepodařilo výšku změřit,
je udávána jako -32768. Tabulka je kódována jako binární soubor, v~němž
jednotlivé řádky zapsány za sebou.
