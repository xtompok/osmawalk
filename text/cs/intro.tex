\chapter{Intro}

\section{Cíl práce}
Cílem práce je specifikace vhodného formátu dat pro vyhledávání pěších tras ve
městě a algoritmů pro přípravu mapových dat do tohoto formátu.  

Cílem práce není vytvořit data pro obecný vyhledávač tras, zaměřili jsme se
pouze na pěší trasy a jejich specifika. Cílem také nebylo vytvořit komplexní
vyhledávací aplikaci, vyhledávací aplikace je pouze ukázkou, jak se dá ve
vytvořených datech hledat.

\section{Zdroje dat}
Našim cílem bylo, aby bylo možné aplikaci i data volně šířit, proto jsme hledali
zdroj mapových dat, který umožňuje jejich volné užití. Nejznámějším a zároveň
nejkvalitnějším projektem, který mapová data poskytuje, je projekt
OpenStreetMap. Tento projekt vytváří mapu světa za pomoci dobrovolníků a také
importuje jiná volně dostupná data. Jedná se o nejkvalitnější volně dostupná
mapová data pro celý svět, proto jsme tento projekt zvolili jako zdroj mapových
dat.

Pro kvalitní hledání potřebujeme i výšková data. Ta používáme ze SRTM, protože
jsou také volně šiřitená a používaná v mnoha jiných mapových projektech.


\section{Programovací jazyky}
Pro implementaci algoritmů v naší práci jsme zvolili Python a C. První část
aplikace připravující z mapy data pro vyhledávání jsme napsali v Pythonu. Tento
objektový skriptovací jazyk jsme zvolili z důvodu předchozích zkušeností a
rychlé tvorby kódu. Jazyk již v základu poskytuje pokročilé datové struktury a
má širokou databázi knihoven usnadňujících vývoj. 

První implementace druhé části přípravné aplikace používaly Python, ale programy
běžely příliš pomalu a byly velmi náročné na paměť. Proto jsme se rozhodli na
druhou část použít jazyk C, který má minimální paměťovou režii a je také velmi
rychlý. Jako doplnění tohoto jazyka jsme užili knihovnu LibUCW, která poskytuje
rychlé implementace složitějších datových struktur (stromy, haldy) a algoritmů
(třídění). 

Pro implementaci aplikace pro vyhledávání trasy jsme také zvolili jazyk C pro
paměťovou nenáročnost a rychlost. Při implementaci jsme se snažili využívat
minimum externích knihoven, aby byla aplikace snadno přenostitelná na jiné
platformy. Grafická nadstavba je napsána v C++ v prostředí Qt, protože toto
prostředí umožňuje snadné vytvoření grafického rozhraní a již pro něj existují
komponenty pro zobrazování mapy. Samotná grafická nadstavba pak používá fukce z
vyhledávače v C.

