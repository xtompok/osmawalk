\chapter{Implementace}
Pro implementaci programu převádějícího OSM data do formátu vyhledávacího grafu
jsme zvolili v první části Python, protože pro něj existuje široké množství
knihoven, které nám pomohly při řešení jednotlivých dílčích problémů. Při
vytváření spojek mezi cestami a jejich následné kontrole již ale nedostačoval,
proto tuto a další části jsme implementovali v jazyce C. Mezi těmito částmi
předáváme data pomocí Protocol Bufferu v souboru. Vyhledávání je také
implementováno v jazyce C kvůli rychlosti a paměťové nenáročnosti.

\section{Použité knihovny a pomocné programy}
Během implementace programu pro přípravu dat jsme se snažili použít co nejvíce
již existujících knihoven a programů pro jednotlivé řešené problémy. Všechny
tyto knihovny a programy jsou nutné pro spuštění a používání programu. 

\medskip
\noindent Používáme tyto pomocné programy:
\begin{itemize}
	\item Výřez s městem z dat pro republiku vyrábíme pomocí 
	{\tuc Osmconvert} (GNU AGPL)\footurl{http://wiki.openstreetmap.org/wiki/Osmconvert}.
	\item Pro práci s Protocol Buffery v Pythonu využíváme třídy generované
	kompilátorem {\tuc protoc} (BSD	New).\footurl{http://code.google.com/p/protobuf/}.
	\item Pro práci s Protocol Buffery v C využíváme funkce a struktury
	generované kompilátorem {\tuc protobuf-c} 
	(BSD 2-Clause)\footurl{https://github.com/protobuf-c/protobuf-c}.
\end{itemize}

\noindent V programech napsaných v Pythonu využíváme následující knihovny:
\begin{itemize}
	\item Na parsování konfiguračních souborů používáme 
	{\tuc PyYAML} (MIT)\footurl{http://pyyaml.org/wiki/PyYAML}.
	\item Na parsování OSM XML používáme {\tuc imposm.parser} 
	(Apache)\footurl{http://imposm.org/docs/imposm.parser/latest/}.
	\item Souřadnice konvertujeme pomocí {\tuc pyproj}
	(MIT)\footurl{http://code.google.com/p/pyproj/}.
	\item Pro hledání komponent grafu využíváme {\tuc networkx} 
	(BSD)\footurl{http://networkx.github.io/}.
\end{itemize}

\noindent V programech napsaných v C využíváme následující knihovny:
\begin{itemize}
	\item Datové struktury, dynamická pole a další potřebné funkce zajišťuje
	{\tuc LibUCW} (GNU LGPL)\footurl{http://www.ucw.cz/libucw/}.
	\item Výpočty se zeměpisnými souřadnicemi provádí {\tuc GeographicLib}
	(MIT)\footurl{http://geographiclib.sourceforge.net/}.
\end{itemize}

Program osmconvert a knihovna GeographicLib jsou již zahrnuty ve zdrojovém kódu,
ostatní knihovny je potřeba zajistit v systému.

\section{Datové struktury}
V průběhu celé přípravy vyhledávacích dat často využíváme několik struktur,
které nyní popíšeme. Datové struktury využívané jen v konkrétních případech
popíšeme u těchto případů.

Při přípravě dat používáme několik {\tuc hešovacích tabulek}. Často potřebujeme
získat uzel respektive cestu s konkrétním identifikátorem, pro tento účel nám
slouží tabulky \verb|nodesIdx| respektive \verb|waysIdx|. V Pythonu jsou
reprezentovány typem slovník, v C využíváme součást knihovny libUCW
\verb|hashtable|. V obou případech je klíčem OSM identifikátor objektu a
hodnotou index do pole uzlů resp. cest, kde je daný objekt uložen.

U každé cesty je v datech uložen seznam uzlů, přes které prochází. Je ale vhodné
mít i pro každý vrchol uložen seznam cest, které jím prochází. Pro tento účel
máme seznam \verb|nodeWays|. Je to seznam seznamů, kdy pod indexem $i$ je seznam
všech identifikátorů cest, které prochází vrcholem na indexu $i$. V Pythonu jde
o seznam seznamů, v C je to rostoucí pole rostoucích polí z libUCW.

Během zpracování hledáme uzly blízké daným uzlů. Abychom pro každé takové
hledání nemuseli procházet všechny uzly, vytvoříme si na začátku {\tuc mřížku}.
Tato mřížka dělí plochu mapy na čtverce $20 \times 20$\,m a v každém čtverci si
uložíme seznam identifikátorů vrcholů, které v něm leží. Poté nám při hledání
sousedů bodu stačí zjistit, do kterého čtverce patří a následně prohledat jen
několik okolních čtverců.

V Pythonu se jedná o třídu \verb|Raster|, která při vytvoření vyrobí mřížku z
mapy. Mřížka je uložena v atributu \verb|raster|, třída má metodu
\verb|getBox(lon,lat)|, která pro bod s  délku \verb|lon| a šířku \verb|lat|
vrátí tuple se souřadnicemi buňky, ve které se bod nachází. Mřížka \verb|raster|
je seznam seznamů seznamů.

V C je mřížka reprezentována strukturou \verb|raster_t|. Samotná mřížka je prvek
\verb|raster| této struktury. V souboru \verb|raster.h| jsou definovány funkce
pro práci s mřížkou. Funkce \verb|makeRaster(map_t map)| vyrobí z mapy mřížku a
vrátí ji. Funkce \verb|getRasterBox(raster_t raster, int64_t lon, int64_t lat)|
dostane jako paramter mřížku, délku a šířku bodu a vrátí pole se dvěma prvky --
souřadnicemi buňky, kde se bod nachází. Mřížka \verb|raster| je reprezentována
rostoucím polem rostoucích polí rostoucích polí z knihovny libUCW.

\medskip


\section{Stažení dat OSM}
Abychom mohli připravovat data pro vyhledávání, musíme nejprve získat data OSM
pro dané město. O tuto činnost se stará skript \verb|prepare.sh|, který stáhne
data pro celou Českou republiku a pomocí osmconvert z ní vyřízne obdélník s
městem. Ten uloží jako \verb|praha.osm|\footnote{Program byl připravován pro
vyhledávání tras po Praze, proto soubory obvykle obsahují název praha.} k
dalšímu zpracování.

\section{Klasifikace dat}
Nejdříve potřebujeme data OSM převést do formátu pro přípravu dat (premap). K
tomuto účelu slouží program \verb|parse.py|, který si načte konfigurační soubory
a podle nich rozdělí jednotlivé uzly a cesty do kategorií. Současně také převede
souřadnice do formátu UTM. Následně smaže všechny uzly, které neleží na žádné
cestě a uloží data do formátu premap.

Konfigurační soubory jsou soubory ve formátu YAML. Používají se následující
konfigurační soubory:
\begin{itemize}
	\item \verb|types.yaml| pro rozdělení cest a multipolygonů do
kategorií 
	\item \verb|area.yaml| pro určení, zda je daný objekt plochou
	\item \verb|tunnel.yaml| pro určení, zda je daný objekt tunelem,
	průchodem či jinou podobnou stavbou
	\item \verb|bridge.yaml| pro určení, zda je daný objekt mostem nebo na
	něm leží
\end{itemize}

Konfigurační soubory mají následující formát:
\begin{verbatim}
barrier: 
    barrier : "*"
    waterway:
        - river
        - canal
water:
    waterway:
        - riverbank
        - stream
\end{verbatim}
Konfigurační soubor se skládá z několika mapování, u každého klíč určuje, jaká
kategorie resp. hodnota se přiřadí, hodnotou mapování první úrovně jsou mapování
druhé úrovně, které určují, za jakých podmínek se hodnota přiřadí. Přiřazuje se
vždy, když je splněna alespoň jedna podmínka. 

V mapování druhé úrovně je klíčem vždy klíč vlastnosti objektu OSM. Hodnota může
být dvou druhů. Buď je to \verb|"*"|, pak stačí, že se shoduje klíč vlastnosti
OSM a na hodnoty se nehledí, nebo je hodnota mapování seznam hodnot objektu OSM.
Pokud má objekt OSM pro daný klíč jednu z těchto hodnot, je  podmínka splněna. 

V příkladu rozdělujeme do kategorií \verb|barrier| a \verb|water|. Pokud má
objekt OSM nějaký atribut s klíčem \verb|barrier|, je tomuto objektu přiřazena
kategorie \verb|barrier| nezávisle na hodnotě tohoto klíče. Obdobně je jako
\verb|barrier| označen objekt OSM, který má atribut s klíčem \verb|waterway| a
hodnotou \verb|river|. Pokud má ale objekt atribut s klíčem \verb|waterway| a
hodnotou \verb|stream|, je klasifikován jako \verb|water|.

Při procházení uzlů převádíme souřadnice do formátu UTM\cite{utmnorma}. Tento
formát používá místo stupňů metry, tudíž se v něm lépe počítají vzdálenosti. To
se nám hodí nejen při přípravě dat, ale i při samotném hledání, protože při
testování se souřadnicemi ve formátu WGS-84 se více času trávilo výpočtem
vzdáleností než zbytkem algoritmu. V našem případě ještě {\tuc vynásobíme souřadnice
UTM desíti}, čímž bude jednotkou 10\,cm což nám jako přesnost stačí a můžeme
všechny výpočty provádět v celých číslech.

Následně vytvoříme hešovací tabulku nodeWays a jejím průchodem zjistíme, které
uzly neleží na žádných cestách a tyto uzly smažeme. Nakonec převedeme data do
formátu premap a uložíme jako soubor \verb|praha-pre.pbf| do složky \verb|data|.

\section{Vyřešení multipolygonů}
\section{Zjednodušení budov}
\section{Rozdělení dlouhých úseků}
\section{Spojky mezi cestami}
\section{Zkratky přes průchozí prostranství}
\section{Vytvoření vyhledávacího grafu}
