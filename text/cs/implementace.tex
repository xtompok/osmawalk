\chapter{Implementace}
Pro implementaci programu převádějícího OSM data do formátu vyhledávacího grafu
jsme zvolili v první části Python, protože pro něj existuje široké množství
knihoven, které nám pomohly při řešení jednotlivých dílčích problémů. Při
vytváření spojek mezi cestami a jejich následné kontrole již ale nedostačoval,
proto tuto a další části jsme implementovali v jazyce C. Mezi těmito částmi
předáváme data pomocí Protocol Bufferu v souboru. Vyhledávání je také
implementováno v jazyce C kvůli rychlosti a paměťové nenáročnosti.

\section{Použité knihovny a pomocné programy}
Během implementace programu pro přípravu dat jsme se snažili použít co nejvíce
již existujících knihoven a programů pro jednotlivé řešené problémy. Všechny
tyto knihovny a programy jsou nutné pro spuštění a používání programu. 

\medskip
\noindent Používáme tyto pomocné programy:
\begin{itemize}
	\item Výřez s městem z dat pro republiku vyrábíme pomocí 
	{\tuc Osmconvert} (GNU AGPL)\footurl{http://wiki.openstreetmap.org/wiki/Osmconvert}.
	\item Pro práci s Protocol Buffery v Pythonu využíváme třídy generované
	kompilátorem {\tuc protoc} (BSD	New).\footurl{http://code.google.com/p/protobuf/}.
	\item Pro práci s Protocol Buffery v C využíváme funkce a struktury
	generované kompilátorem {\tuc protobuf-c} 
	(BSD 2-Clause)\footurl{https://github.com/protobuf-c/protobuf-c}.
\end{itemize}

\noindent V programech napsaných v Pythonu využíváme následující knihovny:
\begin{itemize}
	\item Na parsování konfiguračních souborů používáme 
	{\tuc PyYAML} (MIT)\footurl{http://pyyaml.org/wiki/PyYAML}.
	\item Na parsování OSM XML používáme {\tuc imposm.parser} 
	(Apache)\footurl{http://imposm.org/docs/imposm.parser/latest/}.
	\item Souřadnice konvertujeme pomocí {\tuc pyproj}
	(MIT)\footurl{http://code.google.com/p/pyproj/}.
	\item Pro hledání komponent grafu využíváme {\tuc networkx} 
	(BSD)\footurl{http://networkx.github.io/}.
\end{itemize}

\noindent V programech napsaných v C využíváme následující knihovny:
\begin{itemize}
	\item Datové struktury, dynamická pole a další potřebné funkce zajišťuje
	{\tuc LibUCW} (GNU LGPL)\footurl{http://www.ucw.cz/libucw/}.
	\item Výpočty se zeměpisnými souřadnicemi provádí {\tuc GeographicLib}
	(MIT)\footurl{http://geographiclib.sourceforge.net/}.
\end{itemize}

Program osmconvert a knihovna GeographicLib jsou již zahrnuty ve zdrojovém kódu,
ostatní knihovny je potřeba zajistit v systému.

\section{Příprava dat}

\section{Klasifikace OSM dat}
\section{Vyřešení multipolygonů}
\section{Zjednodušení budov}
\section{Rozdělení dlouhých úseků}
\section{Spojky mezi cestami}
\section{Zkratky přes průchozí prostranství}
\section{Vytvoření vyhledávacího grafu}
