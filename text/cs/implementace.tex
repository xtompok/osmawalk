\chapter{Implementace}
Pro implementaci programu převádějícího OSM data do formátu vyhledávacího grafu
jsme zvolili v první části Python, protože pro něj existuje široké množství
knihoven, které nám pomohly při řešení jednotlivých dílčích problémů. Při
vytváření spojek mezi cestami a jejich následné kontrole již ale nedostačoval,
proto tuto a další části jsme implementovali v jazyce C. Mezi těmito částmi
předáváme data pomocí Protocol Bufferu v souboru. Vyhledávání je také
implementováno v jazyce C kvůli rychlosti a paměťové nenáročnosti.

\section{Použité knihovny}
Během implementace programu pro přípravu dat jsme se snažili použít co nejvíce
již existujících knihoven a programů pro jednotlivé řešené problémy. Popíšeme je
v tom pořadí, v jakém se používají při zpracovávání dat. 
\section{Příprava dat}
\section{Klasifikace OSM dat}
\section{Vyřešení multipolygonů}
\section{Zjednodušení budov}
\section{Rozdělení dlouhých úseků}
\section{Spojky mezi cestami}
\section{Zkratky přes průchozí prostranství}
\section{Vytvoření vyhledávacího grafu}
