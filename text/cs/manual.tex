\chapter*{Příloha A}
\addcontentsline{toc}{chapter}{Příloha A}

V této příloze popíšeme, jak připravit data pro vyhledávání a používat
vyhledávač. Všechny programy napsané přípravu dat jsou zveřejněny pod licencí
MIT. Textová vyhledávací aplikace je zveřejněna také pod licencí MIT, grafická
nadstavba QOsmWalk je licencována pod LGPL, protože jde o odvozené dílo od
projektu QMapControl.

Programy jsou určeny pro operační systém Linux, testovány
byly v distribucích Gentoo a Debian Wheezy. Program pro přípravu dat potřebuje 
alespoň 7\,GB RAM, vyhledávací aplikaci stačí 512\,MB.

\section*{Před prvním spuštěním}
Před prvním spuštěním je potřeba nainstalovat všechny potřebné knihovny a
zkompilovat námi napsané programy. Knihovny nainstalujeme do systému a následně
vstoupíme do adresáře \verb|osm| a zde zadáme \verb|make|. Poté se přesuneme do
složky \verb|compiled| a zde také zadáme \verb|make|. 

Pro zkompilování grafické nadstavby nejprve stáhneme aktuální verzi QMapControl
a složku se souborem \verb|QMapControl.pro| vložíme do kořenové složky. Poté 
otevřeme projekt QOsmWalk v QtDesigneru a sestavíme ho. 

Dále je potřeba nastavit oblast pro generování map a nachystat výšková data. V
souboru \verb|osm/prepare.sh| nastavíme rozsahy zeměpisných šířek a délek a do
složky \verb|osm| nakopírujeme SRTM soubory \verb|.hgt|, které pokrývají
nastavenou oblast.

\section*{Příprava dat}
Při každé aktualizaci dat z OSM je potřeba vykonat následující kroky. Vstoupíme
do adresáře \verb|osm| a spustíme skript \verb|prepare.sh|. Tento skript stáhne
 aktuální mapy, vyřízne z nich používanou oblast a připraví pro ni výšková data.

Následně je potřeba spustit ve složce \verb|scripts/filter/| skripty
\verb|parse.py|, který vytvoří formát pro přípravu dat a \verb|filter.py|, který
provede první část přípravy dat. Po skončení tohoto skriptu se přesuneme do
složky \verb|compiled| a spustíme program \verb|filter|, který dokončí přípravu
dat a vytvoří soubor \verb|data/praha-graph.pbf| s vytvořeným vyhledávacím
grafem.

\section*{Hledání tras}
Výsledky vyhledávacích programů jsou ukládány ve formátu GPX, což je formát pro
popis tras a bodů zájmu podporovaný většinou mapového softwaru. 

Textový vyhledávač je velmi jednoduchý, jako parametry mu předáme souřadnice
výchozího a cílového bodu a program vypíše trasu s informacemi na standardní
výstup a vytvorří soubor \verb|track.gpx| s nalezenou trasou.

Grafický vyhledávač po spuštění umožňuje pohyb v mapě a tlačítky \verb|+| a
\verb|-| mapu
přibližovat a oddalovat. Prvním kliknutím do mapy určíme výchozí bod a druhým
kliknutím určíme cílový bod. Poté je nalezena trasa a zobrazena v mapě. Dalším
kliknutím do mapy určíme nový výchozí bod a původní nalezená trasa je smazána.
Pokud je zobrazena nalezená trasa, je možné kliknutím na tlačítko GPX uložit
nalezenou trasu jako GPX.

