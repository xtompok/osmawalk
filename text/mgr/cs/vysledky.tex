\chapter{Výsledky}
Pro zhodnocení výsledků našeho vyhledávače jsme provedli dva druhy testů. V
rámci prvního jsme porovnávali výsledky našeho vyhledávače s ostatními veřejně
dostupnými vyhledávači spojení. V druhém testu jsme na pevně dané trase a času
zkoumali, jaký vliv na nalezené trasy má různé nastavení penalt a rychlostí
chůze. Aplikaci jsme také profilovali, abychom zjistili, v jakých částech se
tráví nejvíce času a měly by se optimalizovat jako první.

\section{Porovnání s jinými vyhledávači}
Výsledky našeho vyhledávače jsme porovnávali s veřejně dostupnými vyhledávači
spojení po Praze: IDOS\footnote{\url{http://idos.cz}},
Mapy.cz\footnote{\url{http://mapy.cz}} a Google
Maps\footnote{\url{http://maps.google.com}}. 

IDOS je nejstarší a nejznámější vyhledávač spojení. Jako jediný z vyhledávačů
nepodporuje hledání pěších tras, vyhledává pouze v jízdních řádech, přestupy
mezi zastávkami řeší pomocí tabulky, která obsahuje dvojice zastávek a čas
přesunu mezi nimi. Jako vstup je možné zadat adresu, ale vyhledávač nalezne
pouze nejbližší zastávky podle vzdušné vzdálenosti a hledá z nich / do nich. 
Na druhou stranu umožňuje široké možnosti nastavení vyhledávání spojů -- volbu
minimálních časů na přestup, typů dopravních prostředků, počtu přestupů, \dots

Mapy.cz a Google Maps jsou původně mapové služby, které do svých vyhledávačů
přidaly možnost vyhledávání kromě pěších cest i spojení MHD. Protože se jedná
primárně o mapové služby, vyhledávače neumožňují žádné (Mapy.cz) nebo jen velmi
omezené (Google Maps) přizpůsobení. 

Při porovnávání vyhledaných spojení jsme všude nechali výchozí nastavení a náš
vyhledávač jsme nastavili tak, jak si myslíme, že jsou nastaveny ostatní
vyhledávače, aby byly výsledky porovnatelné. Konkrétní nastavení je výchozí
nastavení {\tt config/speeds.yaml} v příloze. Přestupy nebyly nijak
penalizovány, na zastávce bylo nutné mít aspoň 30\,s čas mezi příchodem /
příjezdem a odjezdem.

Testovali jsme následující trasy, které známe i z reálného provozu. Jednotlivé
trasy zkoumaly různé aspekty vyhledávačů:
\begin{itemize}

\end{itemize}
\section{Porovnání různých nastavení}
\section{Profilování}
