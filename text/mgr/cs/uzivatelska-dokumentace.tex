\chapter{Uživatelská dokumentace}
\section{Příprava dat}
Pro přípravu dat je nejprve nutné nainstalovat potřebné externí knihovny do
systému nebo do adresáře ext-lib. Poté je nutné zkompilovat všechny zdrojové
kódy zavoláním {\tt make} v adresáři compiled. Dále je nutné vytvořit databáze
pro přípravu dat. Pro mapová data je potřeba vytvořit databázi osmawalk-prepare,
pro jízdní řády je potřeba vytvořit databázi (\TODO název). Pak je již možné
přistoupit k samotné přípravě dat. Pokud je vše správně nastavené, stačí spustit
skript {\tt prepare.sh} (\TODO implementovat) v kořenovém adresáři projektu a po několika hodinách
budou data připravena. Pokud skript selže, dále popíšeme činnosti, které skript
dělá, aby bylo možné jednotlivé fáze spouštět ručně a hledat, kde nastala chyba.

\begin{enumerate}
\item {\tt osm/prepare.sh} slouží ke stažení dat OSM, SRTM, spojení jednotlivých
dílů dat SRTM a tvorbu výřezu z dat OSM. Výstupem je soubor {\tt osm/praha.osm}
s výřezem z dat OSM a soubor {\tt heights.bin} se spojenými daty SRTM pro daný
výřez. 
\item {\tt compiled/parse} slouží ke klasifikaci dat OSM, určení typů objektů,
určení vrcholů na mostech a pod zemí a přiřazení nadmořské výšky k uzlům.
Výstupem jsou soubory {\tt data/nodes-stage1}, {\tt data/ways-stage1} a {\tt
data/mp-stage1} s jednotlivými objekty. Formát uložení je popsán v sekci Formáty
dat(\TODO ref).
\item {\tt todb.sh} slouží k nahrání klasifikovaných dat do databáze. 
\item {\tt postgis/process.sh} řídí úpravy dat v databázi, uvnitř obsahuje
volání jednotlivých kroků popsaných v kapitole Implementace (\TODO ref).
Výstupem jsou soubory {\tt data/\dots.csv} obsahující vrcholy a hrany dat
připravených pro vyhledávání. 
\item {\tt compiled/csvtograph} Vytváří vyhledávací graf a vybírá v něm největší
souvislou komponentu, kterou uloží pro hledání jako soubor {\tt
data/postgis-graph.pbf}.
\end{enumerate}
\TODO priprava jizdnich radu

\section{Konzolová aplikace}
Konzolová aplikace slouží převážně pro testování, zda vyhledávací knihovna
pracuje stabilně, ale dá se použít i pro jednoduché hledání. Aplikaci spustíme
v adresáři compiled příkzem {\tt ./search <flat> <flon> <tlat> <tlon> [time]},
kde {\tt flat} a {\tt flon} určují zeměpisnou šířku a délku výchozího bodu, {\tt
tlat} a {\tt tlon} souřadnice cílového bodu a {\tt time} unixový timestamp času,
od kterého hledat(\TODO implementovat). Pokud není čas uveden, hledá se od
okamžiku spuštění.  Aplikace najde trasy a na standradní výstup vypíše ke každé
trase použité spoje MHD. Navíc v aktuálním adresáři vytvoří soubory {\tt
track<num>.gpx}, kde {\tt num} značí číslo trasy, které obsahují jednotlivé
nalezené trasy.
 
\section{Webová aplikace}
Webovou aplikaci je možné vyzkoušet na (\TODO odkaz) nebo si ji spustit lokálně
pomocí příkazu {\tt python wsgi.py} v adresáři webapp. V případě, že knihovna
libraptor.so není umístěná tam, kde systém hledá sdílené knihovny, je potřeba ji
do této cesty buď přesunout, nebo zavolat příkaz {\tt export
LD\_LIBRARY\_PATH=<cesta k libraptor.so>}. Po spuštění aplikace je možné si
otevřít prohlížeč na stránce, která se vypíše během spouštění aplikace. I v
případě, že spoušíme aplikaci lokálně, je potřeba mít k dispozici připojení k
internetu, protože mapové podklady a javascriptové knihovny se načítají ze
serverů třetích stran.

Po otevření webové aplikace je zobrazené hlavní okno s mapou. Interakce s mapou,
kromě základního posouvání a zvětšování, probíhá klinutím na bod. Prvním
kliknutím je zvolen výchozí bod, ze kterého hledat, druhým kliknutím je zvolen
cílový bod do kterého se hledá trasa. Volbou cílového bodu je odeslán požadavek
na vyhledání trasy. Po vyhledání jsou v levém sloupci zobrazená shrnutí
jednotlivých nalezených tras v pořadí podle času příjezdu do cíle a první trasa
je zobrazena v mapě. Zobrazení nebo skrytí jednotlivých tras se provede
kliknutím na její souhrn v levém sloupci. Jednotlivé části trasy jsou obarveny
podle typů hran, po kterých vedou a dále jsou zvýrazněna místa přestupu na MHD.
Detailní informace o MHD lze zobrazit kliknutím na ikonu u přestupní zastávky
nebo na hranu spoje MHD. Jednotlivé trasy lze exportovat do GPX kliknutím na
ikonu diskety v souhrnu trasy (\TODO implementovat). 
