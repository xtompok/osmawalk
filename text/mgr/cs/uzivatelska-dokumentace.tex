\chapter{Uživatelská dokumentace}
\section{Příprava dat}
Pro přípravu dat je nejprve nutné nainstalovat potřebné externí knihovny do
systému nebo do adresáře ext-lib. Poté je nutné zkompilovat všechny zdrojové
kódy zavoláním {\tt make} v adresáři compiled. Dále je nutné vytvořit databáze
pro přípravu dat. Pro mapová data je potřeba vytvořit databázi osmawalk-prepare,
pro jízdní řády je potřeba vytvořit databázi (\TODO název). Pak je již možné
přistoupit k samotné přípravě dat. Pokud je vše správně nastavené, stačí spustit
skript {\tt prepare.sh} v kořenovém adresáři projektu a po několika hodinách
budou data připravena. Pokud skript selže, dále popíšeme činnosti, které skript
dělá, aby bylo možné jednotlivé fáze spouštět ručně a hledat, kde nastala chyba.

\begin{enumerate}
\item osm/prepare.sh
\item compiled/parse
\item todb.sh
\item postgis/process.sh
\item compiled/csvtograph

\end{enumerate}
\section{Konzolová aplikace}
\section{Webová aplikace}
Webovou aplikaci je možné vyzkoušet na (\TODO odkaz) nebo si ji spustit lokálně
pomocí příkazu {\tt python wsgi.py} v adresáři webapp. V případě, že knihovna
libraptor.so není umístěná tam, kde systém hledá sdílené knihovny, je potřeba ji
do této cesty buď přesunout, nebo zavolat příkaz {\tt export
LD\_LIBRARY\_PATH=<cesta k libraptor.so>}. Po spuštění aplikace je možné si
otevřít prohlížeč na stránce, která se vypíše během spouštění aplikace. I v
případě, že spoušíme aplikaci lokálně, je potřeba mít k dispozici připojení k
internetu, protože mapové podklady a javascriptové knihovny se načítají ze
serverů třetích stran.

Po otevření webové aplikace je zobrazené hlavní okno s mapou. Interakce s mapou,
kromě základního posouvání a zvětšování, probíhá klinutím na bod. Prvním
kliknutím je zvolen výchozí bod, ze kterého hledat, druhým kliknutím je zvolen
cílový bod do kterého se hledá trasa. Volbou cílového bodu je odeslán požadavek
na vyhledání trasy. Po vyhledání jsou v levém sloupci zobrazená shrnutí
jednotlivých nalezených tras v pořadí podle času příjezdu do cíle a první trasa
je zobrazena v mapě. Zobrazení nebo skrytí jednotlivých tras se provede
kliknutím na její souhrn v levém sloupci. Jednotlivé části trasy jsou obarveny
podle typů hran, po kterých vedou a dále jsou zvýrazněna místa přestupu na MHD.
Detailní informace o MHD lze zobrazit kliknutím na ikonu u přestupní zastávky
nebo na hranu spoje MHD. Jednotlivé trasy lze exportovat do GPX kliknutím na
ikonu diskety v souhrnu trasy (\TODO implementovat). 
