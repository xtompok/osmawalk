\chapter{Vyhledávání trasy}
V připravených datech je možné opakovaně vyhledávat spojení. Ke hledání spojení
se využívá Dijkstrův algoritmus a RAPTOR. Nejprve se naleznou nejbližší vrcholy
grafu k zadaným výchozím a cílovým souřadnicím. Poté se začne Dijkstrovým
algoritmem prohledávat graf pěších cest. Pokud při prohledávání narazíme na
zastávku MHD, provedeme z této zastávky jedno kolo algoritmu RAPTOR, najdeme
tedy všechny zastávky, kam se umíme bez přestupu dostat z dané zastávky. Všechny
tyto zastávky přidáme do fronty Dijkstrova algoritmu s časem dosažení rovným
času příjezdu do dané stanice. 

Při běhu algoritmu počítáme pro každou cestu její penaltu -- čím větší má
cesta penaltu, tím horší pro nás je. Samotný Dijkstrův algoritmus bere vrcholy
podle času dosažení. Po nalezení nejrychlejší cesty se nezastaví, ale počítá
dokud nezpracovává vrchol o nějaký koeficient horší, než je nejkratší cesta. 
Tento postup byl zvolen, protože při hledání cesty se chce uživatel dostat z výchozího
do cílového bodu co nejrychleji, ale má nějaké preference na to, jak by měla
cesta vypadat. Pokud vyhledáváme podle času, dostane uživatel jednak nejkratší
cestu, jednak alternativní cesty, které jsou sice o něco delší, ale více splňují
představy uživatele o optimální cestě.

Pokud bychom během vyhledávání udržovali všechny trojice (vrchol, čas příjezdu,
penalta), byl by procházený stavový prostor obrovský a vyhledávání by bylo velmi
náročné na paměť i čas. Pro redukci stavového prostoru uvažujeme uspořádání pro
každý vrchol, kde spojení $A$ je horší než spojení $B$, pokud $A$ má pozdější
čas dosažení vrcholu a vyšší penaltu než $B$. V takovém případě spojení $A$ do
fronty otevřených vrcholů vůbec nepřidáváme, případně ho zahodíme, když na něj
ve frontě narazíme. Takovéto situaci říkáme, že spojení $B$ majorizuje spojení~$A$

\section{Redukce duplicitních tras}
I přes uvažování pouze nemajorizovaných spojení získáme mnoho spojení, které se
liší jen drobnými změnami v pěších částech trasy. Pro skutečnou cestu podle
nalezeného spojení nemají tyto drobné rozdíly význam, proto se snažíme takovéto
duplicity ignorovat. Základním předpokladem, který učiníme, budiž to, že pokud
se dvě spojení liší v použitém spoji MHD, tak chceme obě zobrazit. Dále se tedy
zabýváme jen spojeními, které mají shodné části využívající MHD. Jako duplicitní
spojení bychom rádi označili taková spojení, která vedou celou cestu blízko
sebe.Takovéto pravidlo je ale časově náročné implementovat, proto místo něj použijeme
zjednodušenou variantu.

Vezmeme pouze pěší části spojení a porovnáváme vždy
odpovídající pěší části proti sobě. Procházíme hrany a hledáme, kde se liší.
Zapamatujeme si index první takovéto hrany. Pak projdeme úsek od konce a opět
hledáme první hranu, kde se liší. Tímto máme ohraničený úsek, kde se cesty liší.
Podíváme se u obou cest na vrchol uprostřed tohoto úseku a pokud jsou tyto
vrcholy příliš blízko sebe, označíme úsek jako duplicitní. Pokud mají dvě
spojení všechny pěší úseky duplicitní, jedno ze spojení zahodíme. Idea je, že
pokud se dvě cesty liší v nějakém úseku, tak uprostřed tohoto úseku budou od
sebe nejdál. Toto pravidlo sice může eliminovat i cesty, které jsou odlišné a
potkávají se zrovna uprostřed, ale to se nestává často a pravidlo lze
implementovat v lineárním čase.
