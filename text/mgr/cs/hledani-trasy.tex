\chapter{Vyhledávání trasy}
V~připravených datech je možné opakovaně vyhledávat spojení. Ke hledání spojení
se využívá Dijkstrův algoritmus a RAPTOR. Nejprve se naleznou nejbližší vrcholy
grafu k~zadaným výchozím a cílovým souřadnicím. Poté se začne Dijkstrovým
algoritmem prohledávat graf pěších cest. Pokud při prohledávání narazíme na
zastávku MHD, provedeme z~této zastávky jedno kolo algoritmu RAPTOR, najdeme
tedy všechny zastávky, kam se umíme bez přestupu dostat z~dané zastávky. Všechny
tyto zastávky přidáme do fronty Dijkstrova algoritmu s~časem dosažení rovným
času příjezdu do dané stanice. 

\section{Penalty}
Samotná nejrychlejší cesta v~síti pěších cest a MHD není vždy ta, kterou chceme
při reálném cestování použít. Navíc vyhledávačů, které takovou cestu i pro
pražskou síť hledají, existuje několik funkčních a široce používaných. Naším
cílem bylo, aby si uživatel mohl parametry nalezené cesty určovat dle svých
preferencí. Proto jsme se snažili navrhnout systém penalt tak, aby se v~něm dalo
vyjádřit co nejvíce běžných situací, kterým bychom se chtěli ve vyhledaném
spojení vyhnout. 

Systém penalt funguje tak, že za každou událost na trase si uživatel může zvolit
nějakou penaltu, která říká, jak moc nerad by danou událost měl na vyhledané
trase. Současně existuje daná mez přípustnosti trasy. Pokud penalta nějaké
částečně vyhledané trasy překročí tuto mez, není již dále pokračováno ve
vyhledávání dál z~této trasy. Tímto zajišťujeme možnost mít jak \uv{měkká}, tak
\uv{tvrdá} kritéria pro hledanou trasu. Také je vhodné mít penalty konstantní a
závislé na době trvání. Například za deště je pro nás čekání na zastávce bez
přístřešku tím horší, čím déle čekáme, zatímco pokud nám vadí cesta eskalátorem
na metro, tak když už ji jednou podnikneme, na délce jízdy metrem už nám
nezáleží.

Důležité je také rozmyslet, jaké všechny události chceme mít možnost
penalizovat:
\begin{itemize}
	\item {\em typ cesty} -- Toto je přímočará myšlenka implementovaná již
v~původním vyhledávání pěších tras. Můžeme jednak penalizovat chůzi po
	silnici, čímž dáme preferenci vyhledávání po chodnících, jednak můžeme
	penalizovat chůzi po nezpevněných pěšinách, například v~zimním období či
	po dešti.
	\item {\em typ průchozího vrcholu} -- Touto událostí je například čekání
	na semaforech, které jsou reprezentovány bodovým znakem na přechodu přes
	silnici. Tvrdým kritériem je například zavřená brána, která spojuje dvě
	volně přístupné oblasti.
\end{itemize}
Další události již souvisí s~cestováním MHD a přestupy:
\begin{itemize}
	\item {\em počet přestupů} -- V~našem případě používáme počet použitých
	vozidel, protože můžeme hledat i pouze pěší trasu. Obvyklé kritérium
	používané i běžnými vyhledávači.
	\item {\em čas na přestup} -- V~základní formě jednoduché
	tvrdé kritérium, které udává minimální čas strávený na zastávce mezi
	příchodem / příjezdem a odjezdem. Toto kritérium lze výrazněji rozšířit
	různými směry. Jednak je možné uvažovat, jak jsme se na zastávku dostali
	a čím z~ní odjíždíme (metro je přesné a spolehlivé, pokud jsme přišli
	pěšky, můžeme popoběhnout, pokud jedeme autobusem, radši si necháme
	větší rezervu), jednak je možné místo fixního času zavést funkci
	závislou na době čekání a tím například preferovat krátká, ale ne moc
	krátká čekání. 
	\item {\em nástup} -- Nejen čas na přestup je možné penalizovat, i
	samotný nástup do vozidla může být penalizován. Základní možností je
	fixní penalta za nástup, která je měkkou variantou omezení počtu
	přestupů. Dále je možné penalizovat určité dopravní prostředky či přestupní
	místa, například zastávky bez přístřešku za deště. 
	\item {\em linka} -- Kromě penalizací dopravních prostředků je také
	vhodné mít možnost penalizovat i linky. Pokud víme, že některá
	autobusová linka je vlivem dlouhé trasy přes město nespolehlivá, můžeme
	ji přiřadit velkou penaltu nebo ji rovnou vyřadit z~vyhledávání. 
\end{itemize}

Další možností rozšíření systému penalt je zapojení denní doby do výpočtu
penalty. Ve večerních hodinách můžeme například penalizovat podchody a jiná
potenciálně nebezpečná místa.

Všechny výše uvedené možnosti jsme při návrhu systému penalt zohlednili.
Základní penalty je možné určit pomocí konfiguračního souboru, pokročilé je
potřeba přidat do kódu sloužícího pro výpočet penalt. Více viz kapitola
\ref{ch:implementace}.

\section{Průběh hledání}
Při běhu algoritmu počítáme pro každou cestu její penaltu -- čím větší má
cesta penaltu, tím horší pro nás je. Samotný Dijkstrův algoritmus bere vrcholy
podle času dosažení. Po nalezení nejrychlejší cesty se nezastaví, ale počítá
dokud nezpracovává vrchol o~nějaký koeficient horší, než je nejkratší cesta. 
Tento postup byl zvolen, protože při hledání cesty se chce uživatel dostat z~výchozího
do cílového bodu co nejrychleji, ale má nějaké preference na to, jak by měla
cesta vypadat. Pokud vyhledáváme podle času, dostane uživatel jednak nejkratší
cestu, jednak alternativní cesty, které jsou sice o~něco delší, ale více splňují
představy uživatele o~optimální cestě.

Pokud bychom během vyhledávání udržovali všechny trojice (vrchol, čas příjezdu,
penalta), byl by procházený stavový prostor obrovský a vyhledávání by bylo velmi
náročné na paměť i čas. Pro redukci stavového prostoru uvažujeme uspořádání pro
každý vrchol, kde spojení $A$ je horší než spojení $B$, pokud $A$ má pozdější
čas dosažení vrcholu a vyšší penaltu než $B$. V~takovém případě spojení $A$ do
fronty otevřených vrcholů vůbec nepřidáváme, případně ho zahodíme, když na něj
ve frontě narazíme. Takovéto situaci říkáme, že {\em spojení $B$ majorizuje
spojení~$A$}.

\section{Redukce duplicitních tras}
I~přes uvažování pouze nemajorizovaných spojení získáme mnoho spojení, která se
liší jen drobnými změnami v~pěších částech trasy. Pro skutečnou cestu podle
nalezeného spojení nemají tyto drobné rozdíly význam, proto se snažíme takovéto
duplicity ignorovat. Základním předpokladem, který učiníme, budiž to, že pokud
se dvě spojení liší v~použitém spoji MHD, tak chceme obě zobrazit. Dále se tedy
zabýváme jen spojeními, které mají shodné části využívající MHD. Jako duplicitní
spojení bychom rádi označili taková spojení, která vedou celou cestu blízko
sebe.Takovéto pravidlo je ale časově náročné implementovat, proto místo něj použijeme
zjednodušenou variantu.

Vezmeme pouze pěší části spojení a porovnáváme vždy
odpovídající pěší části proti sobě. Procházíme hrany a hledáme, kde se liší.
Zapamatujeme si index první takovéto hrany. Pak projdeme úsek od konce a opět
hledáme první hranu, kde se liší. Tímto máme ohraničený úsek, kde se cesty liší.
Podíváme se u~obou cest na vrchol uprostřed tohoto úseku a pokud jsou tyto
vrcholy příliš blízko sebe, označíme úsek jako duplicitní. Pokud mají dvě
spojení všechny pěší úseky duplicitní, jedno ze spojení zahodíme. Idea je, že
pokud se dvě cesty liší v~nějakém úseku, tak uprostřed tohoto úseku budou od
sebe nejdál. Toto pravidlo sice může eliminovat i cesty, které jsou odlišné a
potkávají se zrovna uprostřed, ale to se nestává často a pravidlo lze
implementovat v~lineárním čase.
