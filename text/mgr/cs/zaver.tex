\chapter{Závěr}

\section{Zhodnocení}
Úlohou práce bylo zpracovat veřejně dostupná mapová data a data o~jízdních
řádech a vytvořit z~nich formát vhodný pro vyhledávání spojení kombinujících
pěší přesuny a přesuny hromadnou dopravou. Součástí práce měl být také
vyhledávač, který bude umožňovat nad připravenými daty vyhledávat spojení a bude
možné ho parametrizovat.

Cíl práce se nám podařilo splnit, pro práci s~mapovými daty jsme využili
dřívější bakalářskou práci, kterou jsme dále rozšířili a datový formát
připravili na propojení s~daty z~jízdních řádů. Pro ukládání dat z~jízdních
řádů jsme navrhli formát vycházející z~potřeb algoritmu RAPTOR pro hledání
v~sítích hromadné dopravy. Připravili jsme skripty pro automatický převod dat
jízdních řádů do našeho formátu a implementovali algoritmus RAPTOR jako sdílenou
knihovnu, kterou je možné použít v~jiných projektech nezávisle na zbytku naší
práce. 

Provázali jsme data z~jízdních řádů s~mapovými daty a vypořádali jsme se
s~různou kvalitou zmapování zastávek veřejné dopravy v~mapě. Nad výslednými daty
jsme pak implementovali vyhledávač spojení, který je možné používat jako
sdílenou knihovnu nebo konzolovou či webovou aplikaci. Navrhli jsme systém
hodnocení nalezených tras pomocí penalt a umožnili jsme uživatelům snadno
nastavovat různé penalty pomocí konfiguračního souboru. 

\section{Výsledky}
Spojení nalezená naším vyhledávačem jsou plně srovnatelná se spojeními
nalezenými současnými vyhledávači spojení a na rozdíl od nich můžeme zároveň
využívat možnosti plánovat kvalitní pěší přesuny a nastavovat si parametry
vyhledávače dle svých požadavků. Problémem zůstává nízká rychlost vyhledávání
spojení, víme však, na která místa se zaměřit v~dalších optimalizacích. Návrh
aplikace a převedení přípravy dat do PostgreSQL se osvědčilo a umožnilo snadnou
kontrolu výstupních dat pomocí geografického informačního systému. Rozdělení
přípravy dat do modulů a možnost živého náhledu na zpracovávaná data výrazně
zrychlilo a usnadnilo vyhledávání a odstraňování chyb.

\section{Náměty pro další rozvoj}
Aplikace je v~současné době plně použitelná pro hledání spojení lokálním
uživatelem. Pro použití jako serverová aplikace by kromě zrychlení vyhledávání
bylo nutné i připravit pro uživatele možnost nahrát si vlastní vyhledávací
profily a na straně knihovny se naučit sdílet neměnná mapová data a jízdní řády
mezi více vlákny pro úsporu paměti. 

Zajímavou možností by také bylo upravit vyhledávač do podoby webové aplikace,
optimálně i s~offline daty, a to buď jako samostatnou aplikaci, nebo jako plugin
do některé ze současných mapových aplikací, například OsmAnd.

\subsection{Příprava dat}
Příprava mapových dat je již odladěný proces, který je rychlý a nenáročný na
operační paměť, případné změny by tudíž mířily k~dalšímu vylepšování
připravovaných dat. Jednou z~možností, jak již bylo naznačeno v~kapitole
\ref{ch:vysledky}, by bylo zkoumat okolí cest a mít u~hran v~grafu uložen typ
okolí -- jestli se nacházíme uvnitř zástavby, v~parku či u~velké silnice. Mohly
by tak být penalizovány trasy podle \uv{krásy}.

Přípravu dat jízdních řádů by bylo vhodné přepsat buď jako databázovou aplikaci,
když svá data stejně bere z~databáze, nebo jako samostatnou aplikaci
v~kompilovaném jazyce, která by byla paměťově úspornější. Do výstupního formátu
jízdních řádů by bylo vhodné přidat různé atributy, které se obvykle u~spojů
vyskytují: nízkopodlažnost vozidla, název konečné a další informace, které by
uživatelům usnadnily cestování.

\subsection{Zrychlení vyhledávání}
Pro jiné než lokální použití na počítači je nutné zrychlit vyhledávač spojení.
Zrychlovat vyhledávání je v~současné době možné jak úpravou kódu na
efektivnější, tak změnami v~samotném vyhledávacím algoritmu. Jako první se
nabízí třídit položky u~jednotlivých vrcholů, aby bylo možné majorizace hledat
binárním vyhledáváním místo průchodu celým polem. Z~principu majorizace totiž
u~položek setříděných podle času vzestupně klesá penalta, protože v~opačném
případě by položka s~větším časem a penaltou byla majorizována. Tento přístup by
přinesl složitost do přidávání nových položek, ale zrychlil by rozhodování, zda
položku budeme vůbec přidávat. 

Další možnost zrychlení je přímo ve vyhledávacím algoritmu, kdy můžeme zkoušet
různé jiné typy hald či se pomocí vhodné heuristiky snažit hledat více směrem
k~cíli, jak je uvedeno v~\cite{mj-ga}.

\subsection{Vylepšení vyhledávání}
Mimo zrychlení jsme během práce narazili na různé možnosti, jak zlepšit kvalitu
vyhledávání. Koncept majorizovaných spojení, jak jsme ho navrhli, dává poměrně
dobré výsledky, ale může se snadno stát, že dojde k~majorizaci některých
spojení, které bychom chtěli zachovat. Typická situace je několik možných
začátků trasy, které všechny vedou na stejný spoj. U~nástupního vrcholu do
tohoto spoje se pak sejde několik částečných tras a z~principu penalt bude dále
pokračovat jen jeden z~nich a to ten s~nejmenší penaltou a ostatní se zahodí,
protože mají všechny stejný čas odjezdu. Možným řešením je nemajorizovat jinou
částečnou trasu, která se liší použitými spoji, bylo by však potřeba zjistit,
zda pak vyhledaných spojení nebude příliš mnoho.  

Obdobným problémem je preference častějších spojení, kdy bývá výhodné jít na
zastávku, odkud jezdí spoje často a případný ujetý nebo zpožděný spoj nám
nezkazí celý zbytek cesty. K~tomuto problému je možné přistupovat dvěma způsoby,
jednak lze pro každou vyhledanou trasu zkoumat alternativy v~případě ujetí
jednotlivých spojů po cestě, jednak lze frekvenci spojení nějakým způsobem
zanést do systému penalt a preferovat tak frekventovanější trasy před ostatními. 

Systém penalt lze rozšiřovat mnoha dalšími způsoby, například zavést směrové
penalty. Mezi takové by mohla patřit penalta za chůzi do schodů, ale směrové
penalty dávají smysl u~spojů MHD, kdy poblíž konečné mohou být spoje směrem na
konečnou často zpožděné, zatímco spoje z~konečné bývají spolehlivé a jedou včas.
Dalším vylepšením penaltového systému by byla konfigurace pomocí vhodného
skriptovacího jazyka na místo konfiguračního souboru, kdy by bylo možné si
výpočet penalt přizpůsobit přesně podle svých potřeb a nebylo by pak potřeba
znovu překládat celý program. Tato úprava by ale mohla mít negativní dopad na
rychlost vyhledávání.

Kombinované vyhledávání pěších tras s~přesuny hromadnou dopravou má ještě jedno
specifikum a tím je čas nutného odchodu z~výchozího místa. Pokud jdeme několik
stovek metrů na autobus, který jezdí jednou za půl hodiny, pak není třeba
vycházet hned po nalezení spoje, ale stačí nám vyjít například až za 10 minut a
do cíle dorazíme ve stejný čas. Tato situace je řešitelná poměrně snadno
posunutím pěší části trasy před prvním přejezdem MHD tak, aby pěší trasa na spoj
MHD akorát navazovala. Problém je ale obecnější, pokud autobus s~půlhodinovým
intervalem používáme až na konci cesty, pak nestačí jen posunout první pěší
část, ale mohli bychom jet i pozdější tramvají a tudíž by bylo potřeba i změnit
některé spoje na trase či celou část trasy. Problém se dá (neefektivně) řešit
postupným hledáním v~pozdějších časech, pokud na cestě narazíme na spoj
s~velkým intervalem, ale pro efektivnější řešení by bylo potřeba nastudovat
potřebnou literaturu, provést měření a pravděpodobně by se jednalo o~velký zásah
do vyhledávacího algoritmu.
