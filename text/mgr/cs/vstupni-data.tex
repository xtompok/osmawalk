\chapter{Zdrojová data}

Zdrojová data pro vyhledávač pochází ze dvou zdrojů. Prvním zdrojem jsou mapová
data, která obsahují silnice, cesty, budovy a další mapové prvky. Druhým zdrojem
jsou data o jízdních řádech, která obsahují linky, zastávky a spoje.

\section{Mapová data}
Mapová data jsou použita z projektu OpenStreetMap (OSM). Tato data neobsahují
informace o nadmořské výšce jednotlivých bodů, k doplnění nadmořské výšky jsou
použita data z projektu Shuttle Radar Topography Mission (SRTM), která jsou
lineárně interpolována pro získání nadmořských výšek jednotlivých bodů v mapě.

O datech OSM a jejich problémech jsme již rozsáhle pojednávali v bakalářské
práci (\TODO ref), níže citujeme obecný úvod a popis základních primitiv. 
\newcommand{\tuc}{\bf}
\subsection{Projekt OpenStreetMap}
Projekt OpenStreetMap (\TODO cite{osmweb}) vznikl v~Anglii v~roce 2004 a jeho prvotním cílem bylo
vytvořit volně dostupná geografická data pro Velkou Británii. Iniciativa se
postupně rozrostla do celého světa a dnes mapu pomáhá tvořit přes milion
dobrovolníků. Česká republika je dnes poměrně kvalitně pokryta a zvláště velká
města mají dostatečně detailní pokrytí i pro vyhledávání pěších tras.

\subsection{Datová primitiva OSM} 
Projekt OpenStreetMap používá tři základní geografická primitiva: uzly, cesty a
relace. Ke každému z~těchto primitiv mohou být přiřazeny atributy, což jsou
dvojice klíče a hodnoty. Každý typ primitiv má svou číselnou řadu, ze které
dostává každý prvek unikátní číslo -- id. 
U~polohových dat se neukládá výška, výsledná mapa je pouze dvourozměrná.
Nyní popíšeme jednotlivá primitiva:

{\tuc Uzly} jsou body s~určenými souřadnicemi. Uzly mohou mít atributy, ty pak
určují bodový mapový objekt, například rozcestník nebo závoru. Existují i uzly
bez atributů sloužící pouze jako součást cest nebo relací.

{\tuc Cesty} jsou lomené čáry definované posloupností uzlů. Uzly se na cestě nemohou opakovat
s~jedinou výjimkou: první a poslední bod mohou být shodné, potom se jedná
o~uzavřenou cestu. Cesty jsou orientované, tzn. na pořadí uzlů záleží. Mohou
existovat cesty bez atributů jako součásti relace, ale obvykle mají atributy
určující, jaký objekt reálného světa popisují.

Pomocí cest popisujeme linie a plochy. Pokud má být cesta plochou, musí být
uzavřená, ale ne každá uzavřená cesta je plocha. Tento problém rozebíráme níže.
Jedna cesta také může reprezentovat více fyzických objektů (například silnici
s~tramvajovou tratí, park s~oplocením).

{\tuc Relace} jsou posloupnosti uzlů a cest opatřené atributy. Každý prvek
v~relaci navíc může mít určenou roli. Relace může obsahovat jako prvek i relaci,
ale tato situace není příliš dobře podporována programy pracujícími s~daty OSM. 
Obvykle jsou relacemi reprezentovány složitější objekty, které by se cestami a
uzly popisovaly obtížně, nebo také \uv{virtuální} objekty jako například
trasy linek MHD, cyklotrasy a územní hranice.

Pro nás důležitý typ relace je {\tuc multipolygon}, který se používá k~reprezentaci
složitějších ploch. Plocha reprezentovaná multipolygonem se může skládat z~více
nesouvisejících částí nebo obsahovat díry. Multipolygon obsahuje cesty s~rolemi 
\verb|INNER| resp. \verb|OUTER| indikující, zda je cesta součástí vnější resp.
vnitřní hranice plochy. Hranice multipolygonu se mohou skládat z~více částí, ale
vždy musí dohromady tvořit jednu nebo několik uzavřených částí tvořících obvody 
plochy resp. díry v~ploše.

\section{Výšková data}
O výškových datech jsme také pojednávali v bakalářské práci, opět citujeme (\TODO ref):
Abychom mohli správně odhadnout náročnost pěší trasy, musíme znát i informace
o~nadmořské výšce jednotlivých bodů. Stejně tak jako většina jiných projektů jsme
použili data SRTM (\TODO cite{srtmweb}), což jsou volně dostupná výšková data pro celý
svět.

Shuttle Radar Topography Mission byl projekt NASA, kdy při letu raketoplánu
Endeavour v~roce 2000 byla pomocí radarové interferometrie změřena velká část
Země a zpracovaná data byla následně poskytnuta volně k~dispozici.\footnote{\url{http://dds.cr.usgs.gov/srtm/version2_1/}} 

Výšková data byla měřena po třech úhlových vteřinách (v~USA po jedné vteřině),
tudíž v~České republice jsou data v~mřížce $90\times60$\,m. Data jsou rozdělena
do tabulek po jednom stupni, přičemž sousední tabulky se vždy jedním sloupcem
nebo řádkem překrývají. Každá tabulka má tedy 1201 sloupců a 1201 řádků, kde
řádky odpovídají zeměpisné šířce a sloupce zeměpisné délce.

Nadmořské výšky jsou kódovány celými 16bitovými čísly v~big-endian udávajícími
nadmořskou výšku v~metrech. Pokud se v~některém místě nepodařilo výšku změřit,
je udávána jako $-32768$. Tabulka je kódována jako binární soubor, v~němž
jednotlivé řádky zapsány za sebou.

\section{Jízdní řády}
Data o jízdních řádech jsou očekávána ve formátu General Transit Feed
Specification (GTFS), který je dobře specifikovaný a široce používaný ve světě.
Konkrétně pro testování byla použita data o pražské integrované dopravě od IPR
Praha\footnote{\url{http://opendata.iprpraha.cz/DPP/JR/jrdata.zip}}. Tato data
obsahují metro, tramvaje, autobusy a přívozy v Praze a okolí, bohužel neobsahují
integrované vlakové spoje.

GTFS je distribuováno jako archiv ZIP obsahující jednotlivé tabulky ve formátu
CSV. Význam jednotlivých tabulek je následující:
\begin{itemize}
\item {\em agency.txt} obsahuje informace o dopravcích na jednotlivých linkách.
V našem vyhledávání není používán.
\item {\em stops.txt} obsahuje informace o zastávkách. Zastávky jsou dvou typů.
Prvním typem je zastávkový stojan, který reprezentuje fyzické místo, kde
zastavuje nějaký spoj některé linky. Druhým typem je \uv{stanice}, která udává
oblast několika stojanů, obvykle stejného jména. Nemusí mít fyzickou
reprezentaci a slouží pro zobrazování v mapě a svázání logicky blízkých stojanů.
Protože stanice pro náš vyhledávač nenese důležitou informaci, využíváme pouze
zastávkové stojany (\TODO implementovat). Pro vyhledávání spojení využíváme
následující položky:
\begin{itemize}
	\item {\tt stop\_id} -- jednoznačný identifikátor zastávky
	\item {\tt stop\_name} -- název zastávky
	\item {\tt stop\_lat, stop\_lon} -- poloha zastávky
	\item {\tt location\_type} -- informace, zda jde o stojan, nebo stanici
\end{itemize}
\item {\em routes.txt} obsahuje informace o linkovém vedení. Pro vyhledávání
spojení využíváme následující položky:
\begin{itemize}
	\item {\tt route\_id} -- jednoznačný identifikátor linky
	\item {\tt route\_short\_name} -- krátký název linky, v Praze označení linky
	\item {\tt route\_type} -- typ dopravního prostředku linky, například autobus, loď, metro
\end{itemize}
\item {\em trips.txt} obsahuje informace o spojích. Tato tabulka slouží ke
svázání spoje (cesty konkrétního dopravního prostředku po posloupnosti zastávek
v daný čas), linky a množiny dní, kdy daný spoj jezdí. Pro vyhledávání využíváme
následující položky:
\begin{itemize}
	\item {\tt route\_id} -- identifikátor linky
	\item {\tt service\_id} -- identifikátor jízdních dní spoje
	\item {\tt trip\_id} -- jednoznačný identifikátor spoje
\end{itemize}
\item {\em stop\_times.txt} obsahuje informace o časech zastavení jednotlivých
spojů v jednotlivých zastávkách. Pro vyhledávání spojení používáme následující
položky:
\begin{itemize}
	\item {\tt trip\_id} -- identifikátor spoje
	\item {\tt arrival\_time} -- čas příjezdu spoje do zastávky 
	\item {\tt departure\_time} -- čas odjezdu spoje ze zastávky
	\item {\tt stop\_id} -- identifikátor zastávky
	\item {\tt stop\_sequence} -- pořadí zastávky na spoji. Je potřeba, aby
	bylo nezáporné, celočíselné a po trase spoje se zvyšovalo, zvyšování po
	1 ani počítání od 0 není nutné.
\end{itemize}
\item {\em calendar.txt} obsahuje informace o tom, které dny v týdnu jezdí které
spoje. Pro vyhledávání spojení používáme následující položky:
\begin{itemize}
	\item {\tt service\_id} -- identifikátor jízdních dní spoje
	\item {\tt monday \dots sunday} -- informace, zda spoj jede daný den. 
	\item {\tt start\_date} -- první den platnosti jízdního řádu pro daný
	spoj
	\item {\tt end\_date} -- poslední den platnosti jízdního řádu pro daný
	spoj (tento den ještě v rámci platnosti).
\end{itemize}
\item {\em calendar\_dates.txt} obshauje výjimky z pravidelnosti spojů,
například státní svátky či jiná omezení. Obsahuje změny oproti souboru {\tt
calendar.txt}. Může být použit i samostatně bez souboru {\tt
calendar.txt}, pak obsahuje všechny jízdní dny daného spoje. Pro vyhledávání
spojení používáme následující položky:
\begin{itemize}
	\item {\tt service\_id} -- identifikátor jízdních dní spoje 
	\item {\tt date} -- datum změny
	\item {\tt exception\_type} -- typ změny (1 = jede navíc, 2 = nejede, ač
	by měl)
\end{itemize}
\end{itemize}

