\chapter*{Úvod}
\addcontentsline{toc}{chapter}{Úvod}

Efektivní cestování ve městech bývalo vždy obtížné. S~rostoucí velikostí města
roste i složitost dopravní sítě v~něm a umět se dostat na místo určení rychle
vyžaduje dobrou místní znalost nebo kvalitní vyhledávač a ideálně obojí. Čím
více se člověk cestující po městě vzdaluje od svých vyzkoušených tras, tím více
se musí spolehnout na vyhledávače spojení. Ty ale mohou poskytnout cennou službu
i pro trasy důvěrně známé, zvláště, pokud trasy mají několik alternativ lišících
se podle aktuálního času a návazností po cestě. Od takového vyhledávače je ale
očekáváno, že zvládne naplánovat pěší přestupy i na větší vzdálenosti, než jsou
zastávkové stojany téhož jména a že ho bude možné nastavit dle zkušeností a
preferencí jednotlivých uživatelů. 

V~současné době existuje několik veřejně dostupných vyhledávačů spojení po
městech České republiky: IDOS, Mapy.cz a Google Maps. Žádný z~nich bohužel
zároveň neumožňuje nastavit podrobnější parametry hledané trasy a plánovat
kvalitní pěší přestupy. Rozhodli jsme se proto zpracovat veřejně dostupná mapová
data a data o~jízdních řádech a vytvořit vyhledávač, který by obě zmíněné
podmínky splňoval. S~vyhledávačem také nutně souvisí vytvoření vhodného formátu
dat pro vyhledávání spojení.

V~první kapitole se zabýváme různými známými datovými reprezentacemi sítí hromadné
dopravy. Ve druhé kapitole představujeme mapová data a data o~jízdních řádech,
ze kterých jsme vycházeli a také popisujeme formáty dat, které v~práci
používáme. Ve třetí kapitole popisujeme zpracování zdrojových dat a přípravu
formátů pro vyhledávání spojení, návrhem vyhledávače se zabývá čtvrtá kapitola.
Pátá kapitola je věnována popisu implementace přípravy dat a samotného
vyhledávače, v~šesté kapitole jsou podrobně rozepsány používané datové
struktury. Sedmá kapitola je tvořena dokumentací pro uživatele, kteří by si
chtěli vyhledávač vyzkoušet. V~osmé kapitole popisujeme experimenty, kterými
jsme ověřovali schopnosti vyhledávače v~porovnání s~konkurencí a zkoumali, jaký
vliv na vyhledané trasy budou mít různé konfigurační parametry. Také jsme
zkoumali rychlost vyhledávače a hledali místa, která by pomohla hledání
urychlit. V~závěru shrnujeme dosažené výsledky a navrhujeme možné další úpravy
a rozšíření, kterými by bylo možné vyhledávač dále vylepšit.

V~příloze A~popisujeme obsah přiloženého CD.
