\chapter{Formáty dat}
V rámci přípravy dat a vyhledávání tras jednak používáme interní struktury
jednotlivých programovacích jazyků, jednak některé standardní formáty nezávislé
na programovacím jazyku. Tyto formáty ppředstvíme v následující sekci, v dalších
sekcích se již jen budeme věnovat obsahu jednotlivých struktur.
\section{Používané formáty dat}
\subsection{Protocol Buffers}
Protocol Buffers (PBF) je způsob uložení strukturovaných dat. Byl navržen
Googlem a je používán v případech, kdy potřebujeme přenést po síti zprávy s
danou strukturou. Formát podoporuje různé datové typy (integer různých přesností
a znaménkovosti, čísla s plovoucí řádovou čárkou, stringy, \dots) a
strukturování zpráv. Komunikace probíhá ve fázích naplnění zprávy, její
zabalení, odeslání, přijmutí a rozbalení. Při zabalení jsou data zkompirmována
jednoduchým algoritmem, aby například malá čísla nezabírala zbytečně mnoho
místa. Existují PBF verze 2 a verze 3. V naší práci používáme verzi 2, protože v
době psaní bakalářské práce, kterou v naší práci rozšiřujeme, nebyly PBF verze 3
ještě k dispozici a verze 3 nepřináší novinky, které by motivovaly k přechodu.

Základní jednotkou PBF je zpráva, která obsahuje několik položek. Položkou může
být základní datový typ nebo jiná zpráva, položky mohou být povinné, volitelné,
nebo opakované. Každá položka má jméno, typ a číslo, které ho identifikuje v
binární podobě zprávy. Zprávy se popisují pomocí vlastního jazyka, ze kterého se
pak pomocí kompilátoru vytvoří vazby do jednotlivých programovacích jazyků. 
Podrobnou dokumentaci včetně tutoriálů lze najít na stránkách projektu (\TODO
ref).

\subsection{JavaScript Object Notation}
JavaScript Object Notation(\TODO ref) (JSON) je formát navržený pro předávání dat ve
webových aplikacích. Díky své jednoduchosti a čitelné reprezentaci je hojně
využíván nejen ve webových projektech, ale často i jako konfigurační jazyk.
Vychází z Javascriptu, datové typy v něm jsou kromě primitivních typů slovník --
neuspořádaný seznam dvojic klíč--hodnota a pole -- uspořádaná posloupnost
položek. Formát zpráv není dopředu dán a podpora v programovacích jazycích je
přímo integrovaná nebo dodaná pomocí knihovny. 

\subsection{GeoJSON}
GeoJSON(\TODO ref) je, jak již název napovídá, nadstavba formátu JSON pro přenos
geografických informací. Z hlediska syntaxe se jedná o korektní JSON, který má
ale předepsanou strukturu a názvy položek. Umožňuje ukládat body, linie, plochy,
multipolygony a další objekty spolu s jejich atributy a je nativně podporován
většinou JavaScriptových knihoven pro zobrazování mapových dat.

\section{Formáty používané při přípravě dat}
\TODO PBF
\section{Formáty používané při vyhledávání tras}
\subsection{Formát jízdních řádů}
Jízdní řády jsou po zpracování uloženy ve zprávě {\tt Timetable}. Způsob
provázání dat v jednotlivých položkách je popsán v kapitole Implementace (\TODO
ref). Zpráva má následující položky:
\TODO zjistit, zda v popisu implementace je způsob ukládání dat
\begin{itemize}
	\item {\tt routes} obsahuje popis linek
	\item {\tt stop\_times} obsahuje časy zastavení spojů 
	\item {\tt trip\_validity} obsahuje pro každý spoj index na platnost 
	\item {\tt validities} platnost -- obsahuje popis, které dny nějaký spoj
	jede a které ne 
	\item {\tt route\_stops} udává, která linka má jaké zastávky 
	\item {\tt stops} obsahuje popis zastávek
	\item {\tt transfers} obsahuje popis pěších přestupů, v práci není
	používá 
	\item {\tt stop\_routes} udává, které linky jedou přes jakou zastávku
\end{itemize} 
Každá linka je reprezentována zprávou typu {\tt Route} s následujícími
položkami:
\begin{itemize}
	\item {\tt id} udává RAPTOR id linky 
	\item {\tt nstops} udává počet zastávke linky
	\item {\tt ntrips} udává počet spojů linky
	\item {\tt stopsidx} udává index do pole {\tt route\_stops}, kde
	začínají zastávky linky
	\item {\tt tripsidx} udává index do pole {\tt stop\_times}, kde
	začínají časy spojů linky
	\item {\tt servicesidx} udává index do pole {\tt trip\_validities}, kde
	začínají platnosti spojů linky
	\item {\tt name} udává název linky
	\item {\tt type} udává typ dopravního prostředku
\end{itemize}
Každá zastávka je reprezentována zprávou typu {\tt Stop} s následujícími
položkami:
\begin{itemize}
	\item {\tt id} udává RAPTOR id zastávky
	\item {\tt nroutes} udává počet linek projíždějících zastávkou
	\item {\tt ntransfers} udává počet pěších přestupů ze zastávky
	\item {\tt routeidx} udává index do pole {\tt stop\_routes}, kde
	začínají linky dané zastávky
	\item {\tt transferidx} udává index do pole {\tt transfers}, kde
	začínají přestupy ze zastávky
	\item {\tt name} udává jméno zastávky
	\item {\tt underground} udává, zda je zastávka v podzemí
\end{itemize}
Každé zastavení spoje je reprezentováno zprávou typu {\tt StopTime} s
následujícími položkami:
\begin{itemize}
	\item {\tt arrival} udává čas příjezdu do zastávky
	\item {\tt departure} udává čas odjezdu ze zastávky
\end{itemize}
Každá množina dní, kdy některý spoj jede, je reprezentována pomocí zprávy typu
{\tt Validity}, která má následující položky:
\begin{itemize}
	\item {\tt start} udává UNIX timestamp prvního dne platnosti 
	\item {\tt end} udává UNIX timestamp posledního dne platnosti
	\item {\tt bitmap} udává bitmapu platnosti počínaje prvním dnem
	platnosti. Každý den odpovídá jednomu bitu.
\end{itemize}
Pokud bychom měli nějaké pěší přestupy, byly by reprezentovány zprávou typu {\tt
Transfer} s následujícími položkami:
\begin{itemize}
	\item {\tt from} udává počáteční zastávku
	\item {\tt to} udává koncovou zastávku
	\item {\tt time} udává čas potřebný pro přesun mezi zastávkami
\end{itemize}
\TODO praha-graph.pbf, JŘ pro konkrétní den, struktury pro hledání,

\subsection{Výsledky vyhledávání}
Výsledky vyhledávání jsou z knihovny předávány jako zabalená zpráva PBF. Zpráva
obsahuje pouze jednu opakovanou položku typu {\tt Route}, která obsahuje všechny
nalezené trasy. Každá trasa pak má následující položky: 
\begin{itemize}
	\item {\tt time} udává čas příjezdu do cílového bodu
	\item {\tt dist} udává celkovou délku pěších přechodů na trase
	\item {\tt penalty} udává penaltu trasy
	\item {\tt point} obsahuje jednotlivé body trasy
\end{itemize}
Jednotlivé body trasy jsou spolu s hranou, která do nich vede, popsány zprávou
{\tt Point} s následujícími položkami: 
\begin{itemize}
	\item {\tt lat} a {\tt lon} udávají zeměpisnou šířku a délku bodu v
	systému WGS-84
	\item {\tt height} udává nadmořskou výšku bodu
	\item {\tt departure} pokud je hrana MHD, pak udává čas odjezdu z místa
	nástupu
	\item {\tt arrival} udává čas příchodu/příjezdu do bodu 
	\item {\tt osmvert} pokud je bod z OSM, pak obsahuje informace o vrcholu
	z OSM, je to zpráva typu {\tt graph.Vertex}, která je popsána výše
	\item {\tt stop} pokud je bod zastávka, pak obsahuje informace o vrcholu z GTFS
	\item {\tt edgetype} udává, zda jde o pěší hranu, nebo přesun MHD
	\item {\tt walkedge} pokud je hrana chůze, pak obsahuje informace o
	hraně z OSM, je to zpráva typu {\tt graph.Edge}, která je popsána výše
	\item {\tt ptedge} pokud je hrana MHD, pak obsahuje informace o hraně z
	GTFS.
\end{itemize}
Informace o zastávce jsou uloženy ve zprávě {\tt Stop}, která má následující
položky:
\begin{itemize}
	\item {\tt name} udává název zastávky
	\item {\tt id} udává kód zastávky tak, jak je v GTFS
\end{itemize}
Informace o hraně využívající MHD jsou uloženy ve zprávě {\tt PTEdge}, která
obsahuje následující položky:
\begin{itemize}
	\item {\tt name} udává název linky
	\item {\tt id} udává id linky tak, jak je v GTFS
\end{itemize}
Uvnitř vyhledávací knihovny je výsledek vyhledávání reprezentován typem {\tt
search\_result\_t}, který je svou strukturou velmi podobný výše popsanému PBF.

\TODO GPX

\section{Formáty používané ve webové aplikaci}
\subsection{Výsledky vyhledávání}
Výsledky vyhledávání jsou předávány z backendu do frontendu webové aplikace
(\TODO ref webová aplikace) pomocí formátu JSON. Ten je tvořen polem vyhledaných
tras, kde každá trasa je slovník s následujícími klíči:
\begin{itemize}
	\item {\tt time} udávající čas příjezdu do cíle
	\item {\tt dist} udávající celkovou pěší vzdálenost na nalezené trase
	\item {\tt penalty} udávající penaltu vyhledané trasy
	\item {\tt geojson} obsahující grafickou reprezentaci nalezené trasy. 
\end{itemize}
Grafická reprezentace nalezené trasy je reprezentována ve formátu GeoJSON a
obsahuje lomené linie jednotlivých typů popisující průběh cesty a bodové prvky
pro nástup a výstup z prostředku MHD.

Liniové prvky jsou reprezentovány typem LineString, více sousedících hran
stejného typu je vždy spojeno v jeden úsek typu LineString. Hrany reprezentující různé
spoje MHD se do jednoho úseku nespojují. Úseky mají následující atributy: 
\begin{itemize}
	\item {\tt type} udávající typ úseku (dle číslování objtype)
	\item {\tt name} udávající jméno linky, atribut je přítomen pouze, pokud
	jde o úsek typu MHD
\end{itemize}
Bodové prvky jsou reprezentovány pomocí typu Point s následujícími atributy: 
\begin{itemize}
	\item {\tt type} udávající typ bodu, zde vždy číslo reprezentující objtype PUBLIC\_TRANSPORT  	
	\item {\tt name} udávající jméno zastávky
	\item {\tt subtype} udávající operaci na zastávce ({\tt departure} pro
	nástup, {\tt arrival} pro výstup)
	\item {\tt departure} čas odjezdu spoje ze zastávky
	\item {\tt arrival} pro nástupní bod čas příchodu na zastávku, pro
	výstupní bod čas příjezdu do zastávky
\end{itemize}
\TODO příklad
