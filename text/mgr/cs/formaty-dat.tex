\chapter{Formáty dat}
V rámci přípravy dat a vyhledávání tras jednak používáme interní struktury
jednotlivých programovacích jazyků, jednak některé standardní formáty nezávislé
na programovacím jazyku. Tyto formáty ppředstvíme v následující sekci, v dalších
sekcích se již jen budeme věnovat obsahu jednotlivých struktur.
\section{Používané formáty dat}
\subsection{Protocol Buffers}
Protocol Buffers (PBF) je způsob uložení strukturovaných dat. Byl navržen
Googlem a je používán v případech, kdy potřebujeme přenést po síti zprávy s
danou strukturou. Formát podoporuje různé datové typy (integer různých přesností
a znaménkovosti, čísla s plovoucí řádovou čárkou, stringy, \dots) a
strukturování zpráv. Komunikace probíhá ve fázích naplnění zprávy, její
zabalení, odeslání, přijmutí a rozbalení. Při zabalení jsou data zkompirmována
jednoduchým algoritmem, aby například malá čísla nezabírala zbytečně mnoho
místa. Existují PBF verze 2 a verze 3. V naší práci používáme verzi 2, protože v
době psaní bakalářské práce, kterou v naší práci rozšiřujeme, nebyly PBF verze 3
ještě k dispozici a verze 3 nepřináší novinky, které by motivovaly k přechodu.

Základní jednotkou PBF je zpráva, která obsahuje několik položek. Položkou může
být základní datový typ nebo jiná zpráva, položky mohou být povinné, volitelné,
nebo opakované. Každá položka má jméno, typ a číslo, které ho identifikuje v
binární podobě zprávy. Zprávy se popisují pomocí vlastního jazyka, ze kterého se
pak pomocí kompilátoru vytvoří vazby do jednotlivých programovacích jazyků. 
Podrobnou dokumentaci včetně tutoriálů lze najít na stránkách projektu (\TODO
ref).

\subsection{JavaScript Object Notation}
JavaScript Object Notation(\TODO ref) (JSON) je formát navržený pro předávání dat ve
webových aplikacích. Díky své jednoduchosti a čitelné reprezentaci je hojně
využíván nejen ve webových projektech, ale často i jako konfigurační jazyk.
Vychází z Javascriptu, datové typy v něm jsou kromě primitivních typů slovník --
neuspořádaný seznam dvojic klíč--hodnota a pole -- uspořádaná posloupnost
položek. Formát zpráv není dopředu dán a podpora v programovacích jazycích je
přímo integrovaná nebo dodaná pomocí knihovny. 

\subsection{GeoJSON}
GeoJSON(\TODO ref) je, jak již název napovídá, nadstavba formátu JSON pro přenos
geografických informací. Z hlediska syntaxe se jedná o korektní JSON, který má
ale předepsanou strukturu a názvy položek. Umožňuje ukládat body, linie, plochy,
multipolygony a další objekty spolu s jejich atributy a je nativně podporován
většinou JavaScriptových knihoven pro zobrazování mapových dat.

\section{Formáty používané při přípravě dat}
\TODO PBF
\section{Formáty používané při vyhledávání tras}
\TODO tt.bin, praha-graph.pbf, JŘ pro konkrétní den, struktury pro hledání,
výstupní PBF, GPX
\section{Formáty používané ve webové aplikaci}
\subsection{Výsledky vyhledávání}
Výsledky vyhledávání jsou předávány z backendu do frontendu webové aplikace
(\TODO ref webová aplikace) pomocí formátu JSON. Ten je tvořen polem vyhledaných
tras, kde každá trasa je slovník s následujícími klíči:
\begin{itemize}
	\item {\tt time} udávající čas příjezdu do cíle
	\item {\tt dist} udávající celkovou pěší vzdálenost na nalezené trase
	\item {\tt penalty} udávající penaltu vyhledané trasy
	\item {\tt geojson} obsahující grafickou reprezentaci nalezené trasy. 
\end{itemize}
Grafická reprezentace nalezené trasy je reprezentována ve formátu GeoJSON a
obsahuje lomené linie jednotlivých typů popisující průběh cesty a bodové prvky
pro nástup a výstup z prostředku MHD.

Liniové prvky (\TODO dopsat) 
\begin{itemize}
	\item	
\end{itemize}
Bodové prvky jsou reprezentovány pomocí typu Point s následujícími atributy: 
\begin{itemize}
	\item {\tt type} udávající typ bodu, zde vždy číslo reprezentující objtype PUBLIC\_TRANSPORT  	
	\item {\tt name} udávající jméno zastávky
	\item {\tt subtype} udávající operaci na zastávce ({\tt departure} pro
	nástup, {\tt arrival} pro výstup)
	\item {\tt departure} čas odjezdu spoje ze zastávky
	\item {\tt arrival} pro nástupní bod čas příchodu na zastávku, pro
	výstupní bod čas příjezdu do zastávky
\end{itemize}
\TODO příklad
