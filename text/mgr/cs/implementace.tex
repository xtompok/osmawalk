\chapter{Implementace}
\section{Příprava mapových dat}
Počáteční fáze přípravy dat je shodná s přípravou data v bakalářské práci --
pomocí shellového skriptu je nalezeno poslední vydání OSM dat pro Českou
republiku, tato data jsou stažena, rozbalena a pomocí programu osmconvert je z
nich vyříznuta definovaná oblast. Současně jsou stažena i výšková data z
projektu SRTM. V případě potřeby generování vyhledávacích dat pro jiné město
nebo jinou zemi je potřeba tento skript upravit, pro jiné české město stačí
přepsat hranice výřezu, jiný stát by potřeboval i vlastní zdroj dat OSM a
upravit souřadnice pro stahování SRTM. 

(\TODO nelhat moc, zjistit, co bylo uděláno na SFG) Další pokračování přípravy
dat je sice sémanticky stejné, jako bylo v bakalářské práci, ale mechanizmus
přípravy byl kompletně změněn. Data jsou nejprve klasifikována, pro tento účel
byl vytvořen kód v jazyce C, který dle dodané konfigurace přiřadí typ hranám a
nově i vrcholům. Jazyk C jsme zvolili z důvodu rychlosti a paměťové efektivity,
velkou roli také hrála znalost jazyka autorem. Ve většině programů, které jsme
napsali je také využívána knihovna LibUCW (\TODO ref), která poskytuje velké
množství algoritmů, datových struktur a obvyklých konstrukcí v C a je
naprogramována s ohledem na maximální výkon.

Výsledek klasifikace je uložen do formátu PBF jako objekt Premap. Tento objekt
je následně nahrán do databáze PostgreSQL s nadstavbou PostGIS pomocí kombinace
shellového skriptu a programu v C, který čte Premap a vytváří z něj CSV.
Zkoušeli jsme různé varianty vkládání do databáze, protože jde o velké množství
dat a tento způsob se ukázal nejefektivnější z jednoduchých způsobů vkládání. 

Protože příprava dat je vlastně práce s velkým množstvím objektů naráz, jedná se
o přístup velmi podobný databázovým systémům, které umí s velkým množstvím dat
naráz pracovat. (\TODO někam umístit) 

Další zpracování provádíme v databázi pomocí databázových dotazů a tvorby nových
pomocných tabulek. Oproti zpracování v C, které bylo uvedeno v bakalářské práci,
jsme zpracováním v databázi získali možnost připravovat data i pro velká města
na běžném počítači. Původní verze programu měla totiž všechna data v paměti, což
pro Prahu znamenalo až desítky gigabajtů dat a přepsání programu tak, aby byl
prostorově úspornější by bylo poměrně náročné, byla proto raději zvolena jiná
platforma, která poskytuje několik výhod:
\begin{itemize}
	\item Výměna paměti za čas -- v PostgreSQL si snadno můžeme podle
	dostupné paměti stroje zvolit množství paměti, kterou databáze bude
	využívat. Na strojích s menší velikostí operační paměti se data budou
	připravovat déle, ale zvládnou se připravit. Stejně tak i výkonnější
	stroje mohou být využity na maximum a čas zpracování bude nižší. Všechna
	tato nastavení jsou pouze nastavení databázového systému, na kódu práce
	se nic nemění
	\item Abstrakce -- použitím databázového systému nás odstíní od většiny
	implementačních detailů původní práce a zůstane nám kratší a výstižnější
	kód, který popisuje přípravu dat. Na druhou stranu s použitím
	databázového systému vzniká nutnost tvorby indexů, aby databáze
	pracovala efektivně. Některé jsou přímočaré, u jiných je potřeba zkoumat
	různá řešení, ale použitím nevhodného indexu neztratíme kvalitu dat, jen
	rychlost výpočtu. 
	\item Snadná rozšiřitelnost -- protože jsme se díky abstrakci oprostili
	od implementačních detailů, je mnohem jednodušší upravovat průchod
	přípravou dat, případně do něj přidávat další fáze.
\end{itemize}

\subsection{Rozdělení dlouhých cest}
Rozdělení příliš dlouhých úseků cest jsme implementovali pomocí funkce v jazyku
plpgsql, který je integrován v databázovém systému. Funkce dostane ID cesty ke
zpracování a maximální délku segmentu jako parametry. Pak postupně prochází
jednotlivé segmenty mezi body cesty a v případě, že narazí na segment delší než
je maximální délka segmentu, tak ho postupně od začátku dělí na úseky maximální
povolené délky a přidává do tabulky uzlů uzly, které vznikly dělením. Všechny
uzly (jak původní, tak nově vzniklé) funkce průběžně vrací v tom pořadí, v jakém
se vyskytují na cestě. Z tabulky, která váže uzly k cestám, jsou pak odebrány
staré vazby cest, které byly děleny, a jsou nahrazeny novými vazbami včetně
vložených bodů. Tabulka po této operaci zůstává konzistentní a lze ji použít pro
tvorbu geometrií cest. Dělí se jen pochozí cesty, dělit překážky a
neklasifikované cesty nemá smysl, protože se u nich absence dlouhých úseků
nevyužívá.

\subsection{Tvorba geometrií}
Abychom mohli využívat nadstavby PostGIS pro zodpovídání prostorových dotazů,
musíme nejprve objektům přiřadit geometrie. V datech OSM máme určené pozice bodů
a u složitějších objektů seznam bodů, ze kterých se objekty skládají a jejich
pořadí. Z těchto informací sestavíme pomocí prostředků PostGISu geometrie
jednotlivých objektů a k objektům si je uložíme. Jak jsme již zmínili dříve,
používáme pouze uzly, cesty a z relací multipolygony. Z multipolygonů bereme
pouze vnější okraj, což zařídíme snadno pomocí prostředků PostGISu, který přímo
podporuje multipolygony a umí určit jejich vnější okraj.
 
\subsection{Příprava překážek}
Jako překážky bereme všechny objekty, které klasifikací získaly typ překážka.
U uzavřených cest je potřeba rozhodnout, zda se jedná o obvod nebo i vnitřní
oblast. V rámci klasifikace se u každé cesty na základě kritérií rozhodne,
jestli by daná cesta měla být plochou a tato informace je využívána právě při
přípravě překážek. Součástí tvorby překážek také musí být ošetření nesprávných
geometrií, které vznikají jednak na okrajích mapy, kde některé cesty nejsou
kompletní, protože obsahují body mimo výřez, jednak uvnitř mapy vlivem chybných
editací od uživatelů, kde vznikají špatně uzavřené polygony, dva uzly na stejném
místě a další chyby, které nesmí vést k selhání přípravy dat.

\subsection{Tvorba zkratek}
Tvorba zkratek je stejně jako v bakalářské práci i zde nejsložitější a
nejnáročnější částí přípravy dat. Nejprve je nutné zvolit body, mezi kterými je
možné zkratky vytvářet. Protože musí spojovat nějaké pochozí cesty, nejprve
vybereme všechny vrcholy, které jsou na nějakých pochozích cestách. Zkratky
nelze vytvářet uvnitř překážek ani v podzemí, tyto vrcholy posléze z výběru
odebereme a získáme všechny kandidáty na začátky a konce zkratek. 

Nyní bychom potřebovali najít všechny dvojice bodů, které jsou k sobě blíže než
30\,m (Konstanta byla zvolena empiricky z mapových dat, aby zkratky doplňovaly
potřebné pěší vazby a zároveň nebyly zbytečně dlouhé.) a z nich vybrat některé,
ze kterých se stanou zkratky. Takováto přímočará implementace je ale příliš
pomalá, protože vrcholů je velké množství a funkce pro počítání vzdálenosti je
výpočetně náročná, vede to tedy na dlouhý výpočetní čas.

Abychom tento problém obešli, využili jsme toho, že zkratky jsou lokální a není
tedy třeba porovnávat dvojice bodů, které jsou od sebe příliš vzdáleny. Použili
jsme techniku proložených čtvercových sítích, kde mapový výřez rozdělíme
čtvercovou sítí na čtverce $100\times100$\,metrů, čtverce očíslujeme a každému
bodu přiřadíme číslo čtverce, ve kterém leží. Poté čtvercovou síť posuneme o
50\,metrů uhlopříčně a akci opakujeme. U každého bodu si nyní pamatujeme čísla
dvou čtverců, ve kterých bod leží. Pokud nyní vezmeme libovolné 2 body, které
jsou k sobě blíže než 50\,m, tak budou mít společný první nebo druhý čtverec.
Pokud mají společný první, pak jsme hotovi. Pokud první společný nemají, pak
vezmeme druhý čtverec (\TODO dovysvětlit).

Protože platí předchozí tvrzení, můžeme nyní vzít vždy jen ty dvojice bodů,
které sdílí jeden ze čtverců. Takto jednak získáme méně dvojic, u kterých je
potřeba testovat vzájemná vzdálenost, jednak se takovýto dotaz dá snadno
zrychlit použitím indexů, protože se porovnávají dvě čísla na rovnost. 

Když máme všechny blízké dvojice bodů, jako kandidáty na zkratky uvážíme
všechny úsečky mezi blízkými dvojicemi, které neprotínají žádnou překážku. Z
takovýchto dvojic pak vybíráme zkratky, které skutečně použijeme. Děláme
tak pomocí grafu, kde hrany jsou zkratky a vrcholy konce zkratek. Každou zkratku
pak přidáme s pravděpodobností $\frac{5}{\delta}$, kde $\delta$ je stupeň konce
zkratky s vyšším stupněm v grafu. Tento výběr vede k situaci, že z každého
vrcholu vede v průměru maximálně 5 zkratek, což je empiricky zjištěná konstanta
na základě analýzy mapových dat a experimentů s generováním zkratek.

\section{Příprava jízdních řádů}
Jízdní řády jsou nejprve staženy pomocí shellového skriptu a nahrány do
PostgreSQL databáze, jejíž schema odpovídá formátu GTFS. Průchod přes databázi
jsme zvolili, poněvadž nám pak dává možnost ptát se pouze na část dat,
aniž bychom museli držet celá data v paměti. Výhledově je také možné přesunout
celé předzpracování do databáze, což by ušetřilo paměť nutnou pro předzpracování
v současné implementaci.

Zpracování jízdních řádů dále pokračuje skriptem v jazyce Python, který rozdělí
linky na sublinky, přiřadí k nim spoje a (\TODO dopsat, zmínka o kompirmovaných
tabulkách). Nakonec vytvoří 3 soubory -- {\tt tt.bin} obsahující jízdní řády pro
hledání spojení, {\tt routes.bin} obsahující informace o linkách, v současné
době není využíván a {\tt stopslut.csv} obsahující údaje nutné pro propojení dat
z GTFS s daty z OSM: ID zastávky v GTFS, ID zastávky v připravených datech,
jméno zastávky, její souřadnice a informaci, jestli je zastávka podzemní. 

Data jízdních řádů jsou již připravena pro použití ke hledání, dále zpracováváme
pouze propojovací soubor {\tt stopslut.csv}. Ten nejprve nahrajeme do databáze
jako novou tabulku. Poté zjistíme, které zastávkové stojany mají ID v OSM a tuto
informaci si u zastávek uložíme. Protože následně budeme opět hledat zkratky,
uložíme si pro každou zastávku i údaj o tom, ve kterých dvou čtvercích leží.  
Následuje generování zkratek mezi zastávkami a pochozími cestami. Toto se děje
zvlášť pro povrchové a zvlášť pro podzemní zastávky, povrchové spojujeme s body
na povrchu, podzemní s body v podzemí. Z nalezených zkratek opět vyloučíme ty,
které kříží nějakou překážku a nakonec přidáme vazby stanic metra, které jsou
detailně zmapované v OSM s body zastavení v mapě. Na tuto vazbu používáme
speciální typ hrany, abychom tyto detailní vazby odlišili od obecných zkratek,
které nemusí vždy být možné. 

Nyní již máme v databázi připravená všechna data potřebná pro tvorbu
vyhledávacího grafu. Data z tabulek jsou exportována do CSV a pak je pomocí
jednoduchého programu v C vytvořen výstupní soubor PBF s vyhledávacím grafem a
údaji o vazbě na jízdní řád. V tomto kroku je také kontrolována dosažitelnost
všech vrcholů grafu a je vybrána největší souvislá komponenta, která obsahuje
alespoň polovinu vrcholů. Samotné vyhledávání pak hledá vždy v souvislém grafu.
(\TODO zmínit i v obecném popisu) Protože při změně jízdních řádů dochází ke změnám
očíslování zastávek, je potřeba přegenerovat i tu část generování vyhledávacího
grafu, která zajišťuje párování zastávek. Tato část není výpočetně náročná, je
proto možné ji spouštět při každé změně jízdních řádů. Naopak není přegenerování
nutné, při aktualizaci mapových dat není třeba přegenerovávat jízdní řády. 

\section{Vyhledávání}
Vyhledávací část jsme navrhli jako knihovnu, která umožňuje snadnou integraci do
jiných projektů. Knihovna je napsána v jazyce C s využitím knihovny LibUCW.
Spolu s knihovnou jsme napsali i jednoduchý demonstrační program, který umožňuje
vyhledávání z příkazové řádky. K pohodlnému interaktivnímu hledání jsme napsali
jednoduchou webovou aplikaci, která umožňuje hledání spojení na mapě spolu s
vizualizací jednotlivých nalezených tras a umožňuje jejich export do formátu
GPX.

\subsection{Vyhledávací knihovna}
\subsection{Knihovna algoritmu RAPTOR}

\subsection{Webová aplikace}
Webová aplikace je napsaná v Pythonu ve webovém frameworku Flask(\TODO ref).
Python jsme zvolili z důvodu snadné interakce se sdílenými knihovnami pro C a
rychlosti vývoje aplikací v něm. Flask pak je jednoduchým webovým frameworkem,
který se osvědčil na jiných projektech. Webová aplikace má dvě části -- backend,
který zajišťuje hledání cesty a frontend, který předává informace o zvolených
bodech backendu, zobrazuje mapu a výsledky vyhledávání. 

\subsubsection{Backend}
Backend se kromě renderování hlavní stránky z připraveného templatu stará o
hledání tras a jejich předávání frontendu. Když je ve frontendu zvolen výchozí a
cílový bod, je vyvolán požadavek {\tt GET
/search?flon=<flon>\&flat=<flat>\&tlon=<tlon>\&tlat=<tlat>}, kde {\tt flon} a
{\tt flat} je zeměpisná délka a šířka výchozího bodu a {\tt tlon} a {\tt tlat}
je zeměpisná délka a šířka cílového bodu. Souřadnice výchozího a cílového bodu
jsou předány vyhledávací knihovně a počká se na vrácení výsledku hledání. Tento
výsledek je ve formátu PBF, jak je specifikováno v sekci Formát dat(\TODO ref).
Protože knihovna Leaflet vyžaduje pro zobrazení prostorových dat formát
GeoJSON(\TODO ref), je potřeba nalezené trasy převést do tohoto formátu. Při
tomto převodu jsou současně spojeny všechny navazující hrany stejného typu (v
případě MHD i stejného spoje) do jedné lomené linie a také jsou přidány časy
příchodů na zastávku a odjezdu spoje MHD v případě nástupu do MHD a časy
příjezdů na zastávku v případě výstupu z MHD. Všechna tato geografická data jsou
pak zabalena jako FeatureCollection do formátu GeoJSON a spolu s informacemi o
čase, pěší vzdálenosti a penaltě zabalena do pole všech tras, které je předáno
frontendu ke zpracování. Přesný formát předávaných dat je popsán v sekci Formáty
dat {\TODO ref} 
\subsubsection{Frontend}
Frontend je napsán v Javascriptu s použitím knihovny Leaflet pro operace s
mapou. Jako mapový podklad je použita mapa od MapBoxu (\TODO ref), přes kterou
jsou vykreslovány jednotlivé nalezené trasy.
