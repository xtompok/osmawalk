\section{Zdrojová data}

Zdrojová data pro vyhledávač pochází ze dvou zdrojů. Prvním zdrojem jsou mapová
data, která obsahují silnice, cesty, budovy a další mapové prvky. Druhým zdrojem
jsou data o jízdních řádech, která obsahují linky, zastávky a spoje.

\subsection{Mapová data}
Mapová data jsou použita z projektu OpenStreetMap (OSM). Tato data neobsahují
informace o nadmořské výšce jednotlivých bodů, k doplnění nadmořské výšky jsou
použita data z projektu Shuttle Radar Topography Mission (SRTM), která jsou
lineárně interpolována pro získání nadmořských výšek jednotlivých bodů v mapě.
%TODO Kecy o OSM

\subsection{Jízdní řády}
Data o jízdních řádech jsou očekávána ve formátu General Transit Feed
Specification (GTFS), který je dobře specifikovaný a široce používaný ve světě.
Konkrétně pro testování byla použita data o pražské integrované dopravě od IPR
Praha\footnote{\url{http://opendata.iprpraha.cz/DPP/JR/jrdata.zip}}. Tato data
obsahují metro, tramvaje, autobusy a přívozy v Praze a okolí, bohužel neobsahují
integrované vlakové spoje.

Struktura GTFS je následující:
%TODO rozepsat
\begin{itemize}
\item zastávky
\item linky
\item spoje
\item časy zastavení
\item kalendář
\end{itemize}



\section{Párování zdrojových dat}
Abychom moli plánovat spojení využívající jak pěší chůzi, tak jízdu MHD, je
potřeba data z obou zdrojů vhodně provázat. Máme k dispozici následující údaje:
\begin{enumerate}
\item OSM
\begin{itemize}
	\item jméno zastávky
	\item pozici zastávky
	\item ID zastávky (jen u některých)
\end{itemize}
\item GTFS
\begin{itemize}
	\item jméno zastávky
	\item pozici zastávky
	\item ID zastávky
\end{itemize}
\end{enumerate} 
V ideální připadě by bylo možné spárovat zastávky jednoduše dle ID, bohužel v
OSM má ID jen několik zastávek, většinu zastávek je tedy potřeba spárovat jinak.
Nabízelo by se párování podle pozic a jmen zastávek, ale bohužel zastávky v GTFS
jsou výrazně posunuté oproti OSM i skutečnosti, navíc ne všechny zastávkové
stojany jsou v OSM vyznačeny, zvláště tam, kde je několik zastávkových stojanů
za sebou, například v autobusových terminálech. Pokoušet se párovat zastávky v
GTFS pouze na zastávky v OSM by bylo velmi náročné s nejistým výsledkem.
Využíváme proto toho, že v OSM máme zmapované nejen zastávky, ale i cesty a
zastávkám v GTFS vytváříme speciální vrcholy dle jejich zeměpisné pozice v GTFS
a pomocí zkratek (viz dále TODO ref) je spojujeme s nejbližšími cestami.

Pro zastávky z GTFS, pro které máme v OSM odpovídající ID, použijeme polohu z
OSM a zkratky k cestní síti hledáme z této polohy. Zkratky jsou i zde potřeba,
protože dle pravidel OSM (TODO ref) se zastávka umisťuje na místo, kde zastavuje
vozidlo, což například u tramvají je bod na kolejích, které ale pro pěší
plánování nepoužíváme, tudíž je potřeba najít vhodný blízký bod v cestní síti.

Zvláštní pozornost je potřeba věnovat u párování zastávek metra. V současné
chvíli je metro v Praze zmapované dvěma způsoby. První způsob je novější a
přesnější, jsou při něm zmapována nástupiště a eskalátorové tunely. Při tomto
podrobném mapování jsou také přidána ID stanic, tudíž je možné stanice jednoduše
spárovat a hledat cestu od hrany nástupiště. Častějším způsobem je ale starší
způsob, kdy je stanice metra pouze bod, od kterého vede eskalátorový tunel na
povrch. Tento eskalátorový tunel je pouze virtuální spojka, neodpovídá reálné
poloze podzemních tras. Takovéto stanice rovněž nemají přiřazená ID. U těchto
stanic používáme polohu z GTFS a zkratky spojující zastávku s cestní sítí
hledáme do blízkých míst, která jsou v podzemí, což vede k poměrně dobré
aproximaci přístupu do metra. Jak bude postupovat mapování stanic metra, bude
tento typ stanic postupně eliminován a dojde ke zpřesnění navigace při
přestupech.

Z hlediska implementace jsou vždy preferovány zastávky a stanice zmapované
přesněji, které mají ID, stačí tedy vylepšovat mapu a při dalším předzpracování
dat se nově zmapované zastávky dostanou i do vyhledávače.

\section{Předzpracování dat}
    - parovani dat
    - generovani spojek
\section{Vyhledávání trasy}
    - chodici a mhd hrany
    - funkce na vypocet penalt
    - eliminace duplicitnich tras
