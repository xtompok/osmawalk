\section{Zdrojová data}

Zdrojová data pro vyhledávač pochází ze dvou zdrojů. Prvním zdrojem jsou mapová
data, která obsahují silnice, cesty, budovy a další mapové prvky. Druhým zdrojem
jsou data o jízdních řádech, která obsahují linky, zastávky a spoje.

\subsection{Mapová data}
Mapová data jsou použita z projektu OpenStreetMap (OSM). Tato data neobsahují
informace o nadmořské výšce jednotlivých bodů, k doplnění nadmořské výšky jsou
použita data z projektu Shuttle Radar Topography Mission (SRTM), která jsou
lineárně interpolována pro získání nadmořských výšek jednotlivých bodů v mapě.
%TODO Kecy o OSM

\subsection{Jízdní řády}
Data o jízdních řádech jsou očekávána ve formátu General Transit Feed
Specification (GTFS), který je dobře specifikovaný a široce používaný ve světě.
Konkrétně pro testování byla použita data o pražské integrované dopravě od IPR
Praha\footnote{\url{http://opendata.iprpraha.cz/DPP/JR/jrdata.zip}}. Tato data
obsahují metro, tramvaje, autobusy a přívozy v Praze a okolí, bohužel neobsahují
integrované vlakové spoje.

Struktura GTFS je následující:
%TODO rozepsat
\begin{itemize}
\item zastávky
\item linky
\item spoje
\item časy zastavení
\item kalendář
\end{itemize}

