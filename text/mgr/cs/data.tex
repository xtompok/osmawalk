\chapter{Formáty dat}
\TODO zdrojová data ponížit o 1, dopsat věci o PBF, popisy interních formátů a
JSONu z webove aplikace.
\chapter{Zdrojová data}

Zdrojová data pro vyhledávač pochází ze dvou zdrojů. Prvním zdrojem jsou mapová
data, která obsahují silnice, cesty, budovy a další mapové prvky. Druhým zdrojem
jsou data o jízdních řádech, která obsahují linky, zastávky a spoje.

\section{Mapová data}
Mapová data jsou použita z projektu OpenStreetMap (OSM). Tato data neobsahují
informace o nadmořské výšce jednotlivých bodů, k doplnění nadmořské výšky jsou
použita data z projektu Shuttle Radar Topography Mission (SRTM), která jsou
lineárně interpolována pro získání nadmořských výšek jednotlivých bodů v mapě.
%TODO Kecy o OSM

\section{Jízdní řády}
Data o jízdních řádech jsou očekávána ve formátu General Transit Feed
Specification (GTFS), který je dobře specifikovaný a široce používaný ve světě.
Konkrétně pro testování byla použita data o pražské integrované dopravě od IPR
Praha\footnote{\url{http://opendata.iprpraha.cz/DPP/JR/jrdata.zip}}. Tato data
obsahují metro, tramvaje, autobusy a přívozy v Praze a okolí, bohužel neobsahují
integrované vlakové spoje.

GTFS je distribuováno jako archiv ZIP obsahující jednotlivé tabulky ve formátu
CSV. Význam jednotlivých tabulek je následující:
\begin{itemize}
\item {\em agency.txt} obsahuje informace o dopravcích na jednotlivých linkách.
V našem vyhledávání není používán.
\item {\em stops.txt} obsahuje informace o zastávkách. Zastávky jsou dvou typů.
Prvním typem je zastávkový stojan, který reprezentuje fyzické místo, kde
zastavuje nějaký spoj některé linky. Druhým typem je \uv{stanice}, která udává
oblast několika stojanů, obvykle stejného jména. Nemusí mít fyzickou
reprezentaci a slouží pro zobrazování v mapě a svázání logicky blízkých stojanů.
Protože stanice pro náš vyhledávač nenese důležitou informaci, využíváme pouze
zastávkové stojany (\TODO implementovat). Pro vyhledávání spojení využíváme
následující položky:
\begin{itemize}
	\item {\tt stop\_id} -- jednoznačný identifikátor zastávky
	\item {\tt stop\_name} -- název zastávky
	\item {\tt stop\_lat, stop\_lon} -- poloha zastávky
	\item {\tt location\_type} -- informace, zda jde o stojan, nebo stanici
\end{itemize}
\item {\em routes.txt} obsahuje informace o linkovém vedení. Pro vyhledávání
spojení využíváme následující položky:
\begin{itemize}
	\item {\tt route\_id} -- jednoznačný identifikátor linky
	\item {\tt route\_short\_name} -- krátký název linky, v Praze označení linky
	\item {\tt route\_type} -- typ dopravního prostředku linky, například autobus, loď, metro
\end{itemize}
\item {\em trips.txt} obsahuje informace o spojích. Tato tabulka slouží ke
svázání spoje (cesty konkrétního dopravního prostředku po posloupnosti zastávek
v daný čas), linky a množiny dní, kdy daný spoj jezdí. Pro vyhledávání využíváme
následující položky:
\begin{itemize}
	\item {\tt route\_id} -- identifikátor linky
	\item {\tt service\_id} -- identifikátor jízdních dní spoje
	\item {\tt trip\_id} -- jednoznačný identifikátor spoje
	\item {\tt } -- 
\end{itemize}
\item časy zastavení
\item kalendář
\end{itemize}

