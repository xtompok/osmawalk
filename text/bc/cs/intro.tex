\section{Cíl práce}
Hledání pěších tras v~terénu je velmi rozsáhlé téma s~mnoha specifickými
podoblastmi. Naším hlavním záměrem je hledání tras ve městě,
protože města jsou dostatečně dobře zmapovaná a zároveň pěší trasy z~různých
současných vyhledávačů nebývají optimální. Navíc máme v~plánu práci dále
rozšiřovat a vytvořit vyhledávač spojení po městě využívající kombinaci pěších
tras a městské hromadné dopravy.  Cílem práce je specifikace vhodného formátu
dat pro vyhledávání pěších tras ve městě a algoritmů pro přípravu mapových dat
do tohoto formátu. 

Nechtěli jsme vytvořit data pro obecný vyhledávač tras, zaměřili jsme se
pouze na pěší trasy a jejich specifika. Také jsme se nesnažili vytvořit komplexní
vyhledávací aplikaci; vyhledávací aplikace je pouze ukázkou, že jsou námi
vytvářená data přímo použitelná pro vyhledávání dosahující dobrých výsledků.

\section{Zdroje dat}
Chtěli jsme, aby bylo možné aplikaci i data volně šířit, proto jsme hledali
zdroj mapových dat, který umožňuje jejich volné užití. Nejznámějším projektem,
který mapová data shromažďuje, je projekt OpenStreetMap. Tento projekt vytváří
mapu světa za pomoci dobrovolníků a také importuje jiná volně dostupná data.
Jedná se o~nejkvalitnější volně dostupná mapová data pro celý svět, proto jsme
tento projekt zvolili jako zdroj mapových dat.

Pro kvalitní hledání potřebujeme i výšková data. Ta používáme z~projektu SRTM,
\cite{srtmweb} protože jsou také volně šiřitená a používaná v~mnoha jiných
mapových projektech.

\section{Postup práce}
Problém hledání trasy v~mapě jsme chtěli převést na problém hledání cesty
v~ohodnoceném grafu. Snažili jsme se proto mapová data postupně zpracovat a
vytvořit z~nich graf s~ohodnocenými hranami, ve kterém pak můžeme snadno
vyhledávat například Dijkstrovým algoritmem. Ohodnocení hran však není v~grafu
uvedeno přímo, hrany a vrcholy obsahují informace, ze kterých se  ohodnocení
vypočítává až při vyhledávání. Toto ohodnocení může být ovlivňováno nastavením
vyhledávače.

Při přípravě dat nejprve vybereme rozdělíme mapová data do kategorií a společně
s~výškovými daty je uložíme do formátu pro přípravu dat. Data dále zpracováváme,
převedeme složité objekty na jednodušší, budovy v~blocích nahradíme jejich
obvody, příliš dlouhé úsečky rozdělíme na kratší. V~další fázi mezi blízké cesty
přidáme spojovací hrany, což nám pomůže vypořádat se s~problémem nespojených
cest. Poté přidáme zkratky přes průchozí prostranství a nakonec přípravný formát
převedeme na vyhledávací graf. V~tomto grafu pak vyhledáváme Dijkstrovým
algoritmem.

%%\section{Co už kdo napsal}
%K~porovnání výsledků naší aplikace jsme použili nejznámější webové mapové
%aplikace -- Google Maps a Mapy.cz. Také jsme nalezené trasy porovnávali s~jinými
%vyhledávači používajícími data OSM -- OsmAnd a TODO. Nalezené trasy jsme také
%porovnávali s~vlastní znalostí terénu a skutečně používaných cest.
%

\section{Implementační prostředky}
Pro implementaci algoritmů v~naší práci jsme zvolili jazyky Python a C. První část
aplikace připravující z~mapy data pro vyhledávání jsme napsali v~Pythonu. Tento
objektový skriptovací jazyk jsme zvolili z~důvodu předchozích zkušeností a
rychlé tvorby kódu. Jazyk již v~základu poskytuje pokročilé datové struktury a
má širokou škálu rozšiřujících knihoven usnadňujících vývoj. 

První implementace druhé části přípravné aplikace používaly také Python, ale programy
běžely příliš pomalu a byly velmi náročné na paměť. Proto jsme se rozhodli na
druhou část použít jazyk C, který má minimální paměťovou režii a je také velmi
rychlý. Jako doplnění tohoto jazyka jsme užili knihovnu LibUCW, která poskytuje
rychlé implementace složitějších datových struktur (stromy, haldy) a algoritmů
(třídění). 

Pro implementaci aplikace pro vyhledávání trasy jsme také zvolili jazyk C pro
paměťovou nenáročnost a rychlost. Při implementaci jsme se snažili využívat
minimum externích knihoven, aby byla aplikace snadno přenostitelná na jiné
platformy. Grafická nadstavba je napsána v~C{\tt++} v~prostředí Qt, protože toto
prostředí umožňuje snadné vytvoření grafického rozhraní a již pro něj existují
komponenty pro zobrazování mapy. Samotná grafická nadstavba pak používá fukce
z~vyhledávače v~C.

%%TODO: obrazek - vstupni data a vystupni data

