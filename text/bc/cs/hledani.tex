\chapter{Hledání tras}
Součástí naší práce bylo i vytvoření jednoduchého vyhledávače nad připravenými
daty. Vyhledávač má textový i grafický režim a cestu hledá Dijkstrovým
algoritmem. Rychlosti přesunu po jednotlivých kategoriích hran (viz str.
\pageref{label:kategorie}) můžeme nastavit pomocí konfiguračního souboru.
Program také umožňuje uložit nalezenou trasu jako soubor GPX. \cite{gpxspec}  

\section{Konfigurace}
Vyhledávač umožňuje konfigurovat rychlosti pohybů po hranách jednotlivých
kategorií dvěma způsoby. První je zadáním absolutní rychlosti a druhý je zadáním
poměru k~rychlosti pohybu po obecné cestě. Také je možné konfigurovat penalizaci
za převýšení.

Konfigurační soubor \verb|speeds.yaml| je ve formátu YAML \cite{yamlspec}: 
\begin{verbatim}
speeds:
        WAY: 6
        PARK: 4
        GREEN: 3
ratios:
        PAVED: 1.2
        STEPS: 0.6
heights:
        upscale: 1
        downscale: 1
\end{verbatim}
Konfigurační soubor obsahuje slovník se třemi klíči:
\begin{itemize}
	\item Klíč \verb|speeds| slouží k~nastavení absolutních rychlostí pohybu
	pro jednotlivé kategorie hran. Obsahuje jako hodnotu slovník, jejímiž klíči jsou
	názvy kategorií a hodnotami rychlosti pohybu po cestách dané kategorie
v~kilometrech za hodinu. Vždy je potřeba nastavit rychlost pro kategorii
	\verb|WAY|, z~ní se počítají poměrné rychlosti.
	\item Klíč \verb|ratios| slouží k~nastavení poměrných rychlostí vzhledem
	k~rychlosti kategorie \verb|WAY|. Jeho hodnotou je stejný slovník jako
u~klíče \verb|speeds|, ale jako hodnoty se udávají poměry k~rychlosti
	kategorie \verb|WAY|. Pokud se nějaká kategorie vyskytne ve slovnících pro
	oba klíče \verb|speeds| a \verb|ratios|, má prioritu absolutní rychlost
	z~klíče \verb|speeds|.
	\item Klíč \verb|heights| umožňuje nastavit penalizaci za převýšení.
	Jako hodnotu obsahuje slovník s~klíči \verb|upscale| a \verb|downscale|,
	které udávají, za kolik vodorovných metrů se bude počítat jeden metr
	převýšení při chůzi směrem do kopce a z~kopce. Je možné nastavit kladné,
	záporné hodnoty i nulu, při které se bude převýšení ignorovat. Pokud by
	měla cesta mít po započítání převýšení zápornou délku, je délka
	považována za nulovou.
\end{itemize}

\section{Dijkstrův algoritmus}
Dijkstrův algoritmus \cite{dijkstra} slouží k~nalezení nejkratší cesty mezi dvěma vrcholy grafu.
Algoritmus postupně prohledává graf a zpracovává vrcholy. Vrcholy mohou mít tři
stavy: nedosažený, otevřený a zpracovaný. U~každého vrcholu $v$ také udržujeme
nejkratší dosud známou vzdálenost $d(v)$ z~výchozího vrcholu do daného vrcholu. 
Během procházení grafu si udržujeme množinu otevřených vrcholů. Délku hrany
$(u,v)$ budeme označovat $l(u,v)$.

Na začátku jsou všechny vrcholy nedosažené a jejich vzdálenost je nekonečno.
Označíme výchozí vrchol jako otevřený a přidáme ho do množiny. Jeho vzdálenost
změníme na nulu a opakujeme následující postup:
\begin{enumerate}
	\item Odebereme z~množiny vrchol $v$ s~nejmenší vzdáleností. 
	\item Pokud je $v$ vrchol cílový, našli jsme nejkratší cestu a končíme.
	\item Všem jeho nedosaženým sousedům $u$ nastavíme $d(u)=d(v)+l(v,u)$,
	označíme je jako otevřené a přidáme je do množiny oteřených vrcholů.
	\item Pokud pro nějakého otevřeného souseda $u$ platí $d(u) >
	d(v)+l(u,v)$, pak upravíme $d(u)$ na $d(v)+l(v,u)$.
	\item Vrchol $v$ označíme za zpracovaný.
\end{enumerate}
Pro reprezentaci množiny otevřených vrcholů můžeme výhodně použít haldu, protože
do množiny pouze přidáváme prvky, snižujeme jejich hodnotu a odebíráme z~něj
minimum.

Bylo by možné použít i jiné algoritmy pro hledání nejkratší cesty
\cite{dijkstra}, například A*,  ale Dijkstrův
algortmus je pro hledání ve městě velikosti Prahy dostatečně rychlý a je možno
jej bez problémů použít.


\section{Implementace}
Výkonný kód je uložen v~souboru \verb|searchlib.c|. Jak textový vyhledávač
\verb|search|, tak grafický vyhledávač \verb|QOsmWalk| jsou pouze nadstavby
volající funkce definované v~\verb|searchlib.c|. Program vyžaduje knihovny {\tuc libyaml} a
{\tuc PROJ.4}. Program také používá makra z~{\tuc LibUCW}, ale nepotřebuje být s~touto
knihovnou slinkován. Grafická nadstavba \verb|QOsmWalk| je napsána v~prostředí
Qt.

Soubor \verb|searchlib.c| poskytuje dvě funkce, které pokrývají plně potřeby
vyhledávače. První z~nich je funkce \verb|prepareData|, která načte vyhledávací
graf a konfigurační soubor s~rychlostmi, načež připraví data pro konverzi WGS-84
$\leftrightarrow$ UTM, vytvoří pole \verb|nodeWays|, které pro každý vrchol
obsahuje seznam hran s~ním incidentních a spočítá a uloží délky všech hran. Tuto
funkci stačí zavolat jen jednou při startu programu a uložit si strukturu
s~připravenými daty pro hledání, kterou vrátí.

Druhou z~funkcí je \verb|findPath|, která slouží k~opakovanému hledání pěších
tras. Tato funkce dostane data připravená první funkcí a dvojici zeměpisných
souřadnic. Najde cestu mezi vrcholem v~grafu nejblíže prvním souřadnicím a
vrcholem v~grafu nejblíže druhým souřadnicím. Funkce vrátí strukturu typu
\verb|search_result_t|, která obsahuje pole vrcholů popisující trasu a údaje o~délce
trasy.

Při hledání trasy nejprve převedeme zadané zeměpisné souřadnice do formátu UTM,
ve kterém jsou uloženy souřadnice bodů v~grafu, a následně najdeme nejbližší
vrchol k~zadaným souřadnicím. Poté si připravíme struktury pro Dijkstův
algoritmus a pomocí něj najdeme nejkratší trasu. Nakonec projdeme grafem po
zpětných hranách, vytvoříme pole obsahující procházené body trasy a vrátíme
výsledek.

Abychom mohli spustit Dijkstrův algoritmus, potřebujeme si pro každý vrchol
pamatovat pomocná data. Tato data nemáme uložena přímo v~grafu, ale v~poli
\verb|dijArray|, které má za položky struktury typu \verb|dijnode_t|. Tyto
struktury obsahují následující položky:
\begin{itemize}
	\item \verb|fromIdx| -- ze kterého vrcholu jsme přišli 
	\item \verb|fromEdgeIdx| -- po které hraně jsme přišli
	\item \verb|reached| -- bylo už vrcholu dosaženo? 
	\item \verb|completed| -- byl už vrchol zpracován?
	\item \verb|dist| -- délka nejkratší známé cesty do daného vrcholu
\end{itemize}

Při každé
změně vzdálenosti do nějakého vrcholu si uložíme, z~jakého vrcholu jsme do něj
přišli a po které hraně. Díky tomu pak můžeme projít zpět po těchto uložených
hranách a získáme tím i vrcholy nejkratší cesty. Naší drobnou úpravou je obrácení směru
hledání od cílového vrcholu k~výchozímu. Potom je průchod po zpětných hranách
přímo průchod nejkratší cestou od výchozího bodu k~cílovému. Protože ale
používáme i výšky jednotlivých vrcholů, nejsou hrany v~obou směrech stejně
ohodnocené a proto při hledání od cíle k~výchozímu vrcholu musíme počítat
s~výškami, jako bychom šli v~opačném směru, než jak vede hrana.

Když je cesta nalezena, jen projdeme po zpětných hranách a uložíme nalezenou
cestu do pole. Běhm procházení nalezené cesty převádíme souřadnice vrcholů zpět
z~UTM do WGS-84 a v~této soustavě je také ukládáme v~poli. Každý vrchol má také
uložen v~prvku \verb|type| kategorii hrany, po níž přechází do následujícího
vrcholu. Poslední vrchol má kategorii nastavenu na $-1$.
